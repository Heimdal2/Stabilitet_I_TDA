\section{Introduksjon}

I denne oppgaven setter vi opp et sentralt resultat
i topologisk dataanalyse kalt stabilitetsteoremet.
Teoremet gir et grunnlag for hvorfor TDA (topologisk
dataanalyse) er et nyttig verktøy for å studere
punktskydata fra eksperimenter eller annen type
datainnsamling. Vi lager et topologisk rom fra datasettet
og beregner homologien til rommet. Dette gir oss et
vektorrom for ethvert reelt tall. Det er homologiene og
spesifikt "oppståelsen" og "døden" av et basiselement av
homologiene vi er interesserte i. Vi formulerer livet til
et basiselement som et intervall som starter ved
elementets oppståelse og slutter ved dens død. Mengden av
alle intervallene som beskriver livet til basiselementene
for homologiene er kalt strekkoder og samlingen av
homologiene sammen med overgangsavbildinger mellom
homologiene er kalt varighetshomologien.
Vi definerer en type avstand mellom to varighetshomologier
kalt en innflettingsavstand. Vi definerer også en liknende
avstand på strekkoder som vi også kaller en
inflettingsavstand, vi definerer en annen avstand på
strekkoder kalt flaskehalsavstanden og viser at
innflettingsavstanden og flaskehalsavstanden er like.
Stabilitetsteoremet sier at innflettingsavstanden av
varighetshomologier er lik innflettingsavstanden av deres
korresponderende strekkoder. Oppgaven gir et kategorisk og
mer generalisert perspektiv på stabilitet og
varighetshomologier basert på resultater fra \citep{Bauer2020}.
I stedet for å se på varighetshomologier av datasett ser
vi på samlinger av vektorrom indisert av reelle tall og
overgangsavbildinger mellom dem vi kaller disse
varighetsmoduler. I stedet for strekkoder av
varighetshomologier ser vi på multimengderepresentasjoner
av intervaller som vi også kaller strekkoder. Vi gir en
korrespondans mellom strekkoder og varighetsmoduler og
viser at innflettingsavstanden av en varighetsmodul er lik
flaskehalsavstanden til dens korresponderende strekkode.
