\section{Persistent Homologi}
En grunn til å bry seg om persistensmoduler er fordi de er en generalisering av homologien av en filtrering av et topologisk rom.
\begin{definition}\label{Def:Filt}
    La $X$ være et topologisk rom da er en filtrering på $X$ en følge $F_\bullet X = \{F_tX\}_{t\in\R}$ slik at hvis $s\leq t$ så er $F_sX\subset F_tX$ og $F_\infty X=X$.
\end{definition}
Siden det er en naturlig inklusjon $i_{F_\bullet X}(s,t): F_sX\monicto F_tX$ når $s\leq t$ kan vi se på en filtrering som en funktor $\Rbold\to\Topinc$ hvor $\Topinc$ er kategorien av topologiske rom hvor morfiene er inklusjoner. Vi kan også definere morfier mellom filtreringer:

La $F_\bullet X$ og $G_\bullet X$ være filtreringer av $X$ da er en morfi $f: F_\bullet X\to G_\bullet X$ en følge $\{f(t): F_tX\to G_tX\}$ slik at diagrammet

\[
\begin{tikzcd}
	{F_sX} && {G_sX} \\
	\\
	{F_tX} && {G_tX}
	\arrow["{f(s)}", from=1-1, to=1-3]
	\arrow["{i_{F_\bullet X}(s,t)}", from=1-1, to=3-1]
	\arrow["{i_{G_\bullet X}(s,t)}", from=1-3, to=3-3]
	\arrow["{f(t)}", from=3-1, to=3-3]
\end{tikzcd}
\]
kommuterer. Da er filtreringer av et rom en kategori som vi kaller $\Filt{X}$.

Akkurat som topologiske rom kan vi ta homologien på filtreringer ved komposisjonen $\Rbold\xto{F_\bullet X}\Topinc\xto{H_i}\vect$. Eksplisitt blir dette følgende:

Gitt et rom $X$ la $F_\bullet X$ være en filtrering. Da er $H_i(F_\bullet X)$ en persistensmodul definer ved
\[H_i(F_\bullet X)_t = H_i(F_tX)\]
med overgangsavbildinger
\[\p_{H_i(F_\bullet X)}(s,t) = (i_{F_\bullet X}(s,t))_*\]

Dette er metoden man bruker i topologisk dataanalyse for å studere "formen" på en punktsky av data, noe vi kommer tilbake til i anvendelsene. Før vi kan diskutere slike anvendelser må vi først vite hvordan vi lager topologiske rom ved en punktskyer.

\subsection{Topolgiske rom fra punktskyer}
En punktsky $P\subset\R^d$ er en diskret mengde av punkter i $\R^d$. Dette kan være data om farger eller gråtoner på bilder, nerver i en hjerne osv. 

Det er ikke mye topologisk informasjon vi kan få ut a skyen i seg selv gitt at den er en diskret mengde, men vi kan lage simplisialkomplekser av skyen. Dette kan gjøres på mange måter, men det er to hovedmetoder å gjøre dette på.

\subsubsection{Cech-komplekser}
En måte å lage simplisialkomplekser av en punktsky er ved å lage en $k$-simpleks mellom $k+1$ punkter hvis snittet av $\eps$-ballene i punktene snitter hverandre.

\begin{definition}\label{Def:Cech}
    La $P\subset\R^d$ være en punktsky vi definerer Cech-komplekset ved
    \[\Ccal_\eps(P) = \bset{(x_i)_i \;\mid\; \bigcap_i\bar{\Bcal}(x_i,\eps)\neq\es}.\]
\end{definition}
Problemet med dette komplekset er at en må telle hvor mange sirkler som snitter hverandre.

%%% !!! Her trenge eg å snakka med Morten!!! %%%
\begin{example}\label{Ex:label}
	La $P = \{(-1,1), (1,1), (1,-1), (-1,-1)\}$ for forskjellige verdier av $\eps$ får vi forskjellige simplisialkomplekser gitt her, for å forkorte mengden av simplisialkompleksene skriver vi $v_0=(-1,-1),v_1=(1,-1),v_2=(1,1),v_3=(-1,1)$ i stedet for punktene
	\begin{itemize}
		\item Når $0\leq\eps<\frac{1}{2}$ så er \[\Ccal_\eps(P) = \{[v_0],[v_1],[v_2],[v_3]\}.\]
		\item Når $\frac{1}{2}\leq\eps<\sqrt{2}$ så er \[\Ccal_\eps(P) = \{[v_0],[v_1],[v_2],[v_3],[v_0,v_1], [v_0,v_3], [v_2,v_3], [v_1,v_2]\}.\]
		\item Når $\eps\geq\sqrt{2}$ så er \[\Ccal_\eps(P) = \{[v_0],[v_1],[v_2],[v_3],[v_0,v_1],[v_0,v_2],[v_0,v_3],[v_1,v_2],[v_1,v_3],[v_2,v_3],[v_0,v_1,v_2],[v_0,v_1,v_3],[v_0,v_2,v_3]\}\]
	\end{itemize}
\end{example}

\subsection{Rips-komplekser}
En annen måte å få et simplisialkompleks av en punktsky er å lage et $k$-simpleks mellom $k+1$ punkt hvis de er $\eps$ nærme hverandre.
\begin{definition}\label{Def:Rips}
    La $P\subset\R^d$ være en punktsky, da er Rips-komplekset definert ved
    \[\Rcal_\eps(P) = \bset{(x_i)_i\;\mid\; |x_i-x_j|\leq\eps}\]
\end{definition}

Cech- og Rips-kompleksene er begge filtreringer av simplisialkomplekset der hver kombinasjon av punkter i punktskyen har en simpleks. Dermed er de også filtreringer av topologiske rom. Dette gir oss persistensmodulene
\[H_i(\Ccal_\bullet(P)) = \{H_i(\Ccal_\eps(P))\}_{\eps\in\R}\]
og
\[H_i(\Rcal_\bullet(P)) = \{H_i(\Rcal_\eps(P))\}_{\eps\in\R}.\]
Det er disse man bruker når man studerer de topologiske egenskapene til data.