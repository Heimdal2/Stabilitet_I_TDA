\section{Barkoder}
En barkode $\Bcal$ er en representasjon av en multimengde av intervaller. Elementer i en barkode er dermed par
$(I,k)$ der $I$ er et intervall og $k\in\Nbb$. Ofte når indeksen $k$ ikke er nødvendig skriver vi bare $I$ for et intervall i barkoden.

I \cite{Bauer2015a} sier forfatter at gitt en persistensmodul $M$ som kan skrives
\[M\cong\bigoplus_{I\in\Bcal_M}C(I)\]
Da er $\Bcal_M$ unikt bestemt. Vi kaller slike persistensmoduler intervalldekomponerbare.

Dette er en konsekvens av følgende teorem

\begin{teorem}\label{Thrm:label}
    Hvis $\bigoplus_{I\in\Bcal}C(I)\cong\bigoplus_{J\in\Ccal} C(J)$ så er $\Bcal\cong\Ccal$
\end{teorem}
\bevis{
Anta at det ikke eksisterer en bijeksjon mellom $\Bcal$ og $\Ccal$ f.eks. 
}

I følge \cite{Bauer2015a} har vi følgende teorem
\begin{teorem}\label{Thrm:thrm2.1}
Enhver p.e.d. persistensmodul er intervalldekomponerbar.
\end{teorem}
\bevis{
La $M$ være en p.e.d. peristensmodul. Mengden $I_n = \{s\in\Rbb\;\mid\;\dim M_s=n\}$ er en disjunkt union av intervaller $I_{n_1}\sqcup I_{n_2}\sqcup\dots$, vi kan la $C(I_{n_1}\sqcup I_{n_2}\sqcup\dots)=C(I_{n_1})\oplus C(I_{n_2})\oplus\dots$ og få
\[M\cong \bigoplus_{n,k\in\in\Nbb} C(I_{n_k}).\]
Da blir $\Bcal_M = \{I_{n_j}\;\mid\; n,j\in\Nbb\}$, en kan la $I_{n_j}=\es$ når den ikke påvirker unionen $I_n$. 
}

Barkoder kan også gjøres om til en kategori med morfier
kalt overlappkoblinger. For å definere en  overlappkobling
må vi først definere hva det betyr at et intervall
overlapper et annet intervall over og hva en parvis
kobling av mengder er.
\pagebreak
\begin{definisjon}
    Et intervall $I$ overlapper et annet intervall $J$
    over (hhv. overlapper $J$  $I$ under) hvis følgende
    holder
    \begin{itemize}
        \item $I\cap J\neq\es$.
        \item For enhver $s\in J$ eksisterer det en $t\in I$ slik at $t\leq s$. Vi sier at $I$ begrenser $J$ over.
        \item For enhver $s\in I$ eksisterer det en $t\in J$ slik at $t\leq s$. Vi sier at $J$ begrenser $I$ under.
    \end{itemize}
\end{definisjon}

\begin{definisjon}\label{def:Parvis-Kobling}
  En parvis kobling mellom mengder $S$ og $T$ (skrevet
  $\sigma:S\to T$) er en bijeksjon $\sigma: S'\to T'$
  mellom delmengder $S'\subset S$ og $T'\subset T$.
  Formelt er $\sigma\subset S\times T$ en relasjon slik at
  $(s,t)\in \sigma$ hvis og bare hvis $s\in S'$ og
  $\sigma(s)=t$. Komposisjonen av to parvis koblinger
  $\sigma: S\to T$ og $\tau: T\to U$ er definert som
  relasjonen
  \[\tau\circ\sigma=\bset{(s,u)}{(s,t)\in \sigma, (t,u)\in \tau\;\;\text{for en $t\in T$}}\]
\end{definisjon}

\begin{definisjon}\label{def:OverlappMch}
  En parvis kobling $\sigma: \Ccal\to \Dcal$ av barkoder
  er en overlappkobling hvis gitt $\sigma(I)=J$ så
  overlapper $I$ $J$ over.
\end{definisjon}

Komposisjonen av overlappkoblinger som parvis koblinger
resulterer ikke alltid i en overlappkobling. Vi endrer
komposisjonen på følgende måte

\begin{definisjon}\label{def:OK-Komp}
    La $\sigma: \Bcal\to \Ccal$ og $\tau:\Ccal\to\Dcal$
    være overlappkoblinger. Komposisjonen blir da
    definert som
    \[\tau\bullet\sigma
      = \bset{(I,J)\in\tau\circ\sigma}{\text{$I$
    overlapper $J$}}\]
    Her er $\tau\circ\sigma$ komposisjonen som parvis
    koblinger.
\end{definisjon}

\begin{proposisjon}\label{prop:ID-OK}
For enhver barkode $\Dcal$ eksisterer det alltid en
overlappkobling $\sigma:\Dcal\to\Dcal$ gitt ved $\sigma
= \bset{(I,I)}{I\in\Dcal}$.
\end{proposisjon}
\bevis{
Første punktet er å vise at $I\cap\sigma(I)\neq\es$. Dette
er sant siden $\sigma(I)=I$ altså er $I\cap\sigma(I)=I\cap
I$ som er lik $I\neq\es$. Andre punktet er å vise at $I$
overlapper seg selv over. Velg en $s\in I$ da har vi
$s\leq s$ dermed begrenser $I$ seg selv over og under.
Dette betyr at $\sigma$ er en overlappkobling.
}

\begin{teorem}
    Klassen av barkoder sammen med overlappkoblinger
    skaper en kategori.
\end{teorem}
\bevis{
Objektene er barkodene og morfiene er overlappkoblingene.
Vi lar identiteten til en barkode være overlappkoblingen
$\id_\Dcal:\Dcal\to\Dcal$ gitt
i \bref{proposisjon}{prop:ID-OK} Komposisjonen er den
definert i \bref{definisjon}{def:OK-Komp}. Vi må vise at
$\bullet$ komposisjonen er assosiativ og respekterer
identiteter.

Vi starter med identitetene. La $\sigma:
\Dcal\to\Ccal$ være en overlappkobling da har vi
\[\sigma\bullet\id_\Ccal = \bset{(I,J)\in\sigma\circ\id_\Ccal}{\text{$I$ overlapper $J$ over}}.\]
La $(I,J)\in \sigma\bullet\id_\Ccal$, Da eksisterer det en
$K\in\Ccal$ slik at $(I,K)\in\id_\Ccal$ og
$(K,J)\in\sigma$ og $I$ overlapper $J$ over. Siden
$\id_\Ccal(I)=I$ så må $K=I$ altså er
$\sigma(K)=\sigma(I)=J$, derfor har vi at
$(I,J)\in\sigma$. Dette betyr at
$\sigma\bullet\id_\Ccal\subset\sigma$ og gitt et par
$(I,J)\in\sigma$ har vi paret $(I,I)\in\id_\Ccal$ slik at
$J = \sigma(I) = \sigma\bullet\id_\Ccal(I)$ dermed har vi
også at $\sigma\subset\sigma\bullet\id_\Ccal$ det vil si
at $\sigma=\sigma\bullet\id_\Ccal$. På samme måte kan vi
vise at $\sigma = \id_\Dcal\bullet\sigma$.

La $\sigma:\Bcal\to\Ccal$, $\tau:\Ccal\to\Dcal$ og
$\psi:\Dcal\to\Ecal$. La $(I,K)$ være et par
i komposisjonen $(\psi\bullet\tau)\bullet\sigma$. Dette
betyr at det eksisterer et intervall $J\in\Ccal$ slik at
$(I,J)\in\sigma$ og $(J,K)\in\psi\bullet\tau$ og at $I$
overlapper $K$ over. Siden $(J,K)\in\psi\bullet\tau$ må
det også finnes et intervall $L\in\Dcal$ slik at
$(J,L)\in\tau$ og $(L,K)\in\psi$ og $J$ må overlappe $K$
over. Fordi $(I,J)\in\sigma$ og $(J,L)\in\tau$
}

\begin{proposisjon}\label{prop:Ønull}
  Den tomme mengden $\empty$ er nullobjektet i $\Barc$. 
\end{proposisjon}
\bevis{
  La $\Dcal$ være en barkode. En overlappkobling $\sigma:
  \empty\to\Dcal$ er en delmengde av
  $\empty\times\Barc=\empty$, siden $\empty$ er sin eneste
  delmengde så er $\sigma=\empty$. For samme grunn er det
  bare en overlappkobling $\Dcal\to\empty$. Dermed er
  $\empty$ nullobjekt.
}

\begin{teorem}\label{trm:ker}
  Kjernen til en overlappkobling $\sigma:\Ccal\to\Dcal$
  er en barkode $\ker\sigma$ med overlappkobling
  $\kappa_\sigma:\ker\sigma\to\Ccal$
  \[\ker\sigma = \bbag{\ker(\sigma,
    I)\neq\empty}{I\in\Ccal},\quad\kappa_\sigma
  = \bset{(\ker(\sigma,I),I)}{\ker(\sigma,I)\neq\empty}\]
\end{teorem}
\bevis{
  Gitt en barkode $\Bcal$ og en overlappkobling
  $\eta:\Bcal\to\Ccal$ slik at $\sigma\bullet\eta=\empty$,
  viser vi at det eksisterer en unik overlappkobling
  $\eta':\Bcal\to\ker\sigma$ slik at
  $\eta=\kappa_\sigma\bullet\eta'$. Vi antar at en slik
  $\eta'$ eksisterer og finner et uttrykk for
  overlappkobling. La $(I,J)\in\eta$ altså er
  $(I,J)\in\kappa_\sigma\bullet\eta'$. Fra
  \bref{definisjon}{OK-Komp} betyr dette at det
  eksisterer et intervall $K\in\Ccal$ slik at
  $(I,K)\in\eta'$ og $(K,J)\in\kappa_\sigma$. Siden
  $(K,J)\in\kappa_\sigma$ må $K = \ker(\sigma,J)$. Dermed
  har vi
  \[\eta'(I) = \ker(\sigma,\eta(I)).\]
  Her lar vi $\eta'$ koble $I\in\Bcal$ hvis $\eta$
  kobler $I$ og $I$ overlapper $\ker(\sigma,\eta(I))$
  over. Dermed for ethvert par
  $(\Bcal,\eta:\Bcal\to\Ccal)$ av en barkode og
  overlappkobling slik at $\sigma\bullet\eta=\empty$,
  eksisterer det en unik overlappkobling
  $\eta':\Bcal\to\Ccal$ definert som over slik at
  $\kappa_\sigma\bullet\eta' = \eta$.
}

\begin{teorem}\label{trm:coker}
  Kokjernen til en overlappkobling er gitt som en
  barkode $\coker\sigma$ med en overlappkobling
  $\mu_\sigma:\Dcal\to\coker\sigma$ gitt ved
  \[\coker\sigma
  = \bbag{\coker(\sigma,J)\neq\empty}{J\in\Dcal},\quad
  \mu_\sigma = \bbag{(J,\coker(\sigma,J))}{\text{$J$
  overlapper $\coker(\sigma,J)$ over}}\]
\end{teorem}
\bevis{
  Vi viser at gitt en barkode $\Bcal$ og en
  overlappkobling $\eta:\Dcal\to\Bcal$ slik at
  $\eta\bullet\sigma=\empty$ eksisterer det en unik
  overlappkobling $\eta':\coker\sigma\to\Bcal$ slik at
  $\eta'\bullet\mu_\sigma=\eta$. Vi gjør dette på samme
  måte som i beviset for \bref{teorem}{trm:ker}. La
  $(I,J)\in\eta$ dermed er
  $(I,J)\in\eta'\bullet\mu_\sigma$. Siden
  $(I,J)\in\eta'\bullet\mu_\sigma$ eksisterer det et
  intervall $K\in\coker\sigma$ slik at
  $(I,K)\in\mu_\sigma$ og $(K,J)\in\eta'$. Siden
  $(I,K)\in\mu_\sigma$ er $K = \coker(\sigma,I)$. Dermed er
$(\coker(\sigma,I),J)\in\eta'$. Vi lar dermed
$\eta'(\coker(\sigma,I)) = J$ hvor $(I,J)\in\eta$ og
$\coker(\sigma,I)$ overlapper $J$ over.
}

\begin{teorem}\label{trm:im}
  Bildet av en overlappkobling $\sigma:\Ccal\to\Dcal$ er
  en barkode $\im\sigma$ og en overlappkobling
  $\omega_\sigma:\im\sigma\to\Dcal$ gitt ved
  \[\im\sigma = \bbag{I\cap J}{(I,J)\in\sigma},\quad
  \omega_\sigma = \bbag{(I\cap J, J)}{(I,J)\in\sigma}\]
\end{teorem}
\bevis{
  Siden $\Barc$ er en Puppe-eksakt kategori er bildet av
  $\sigma$ kjernen av kokjernen av $\sigma$. Barkoden blir
  multimengden av elementer på formen
  $J-(\coker(\sigma,J))\neq\empty$. Hvis $\sigma$ ikke
  parvis kobler $J$ er $J-(\coker(\sigma,J))=J-J=\empty$,
  hvis $(I,J)\in\sigma$, er $J-(\coker(\sigma,J))=J-(J-I)$
  som er det samme som $I\cap J$. Siden $(I,J)\in\sigma$
  overlapper $I$ $J$ over, altså er $I\cap J\neq\empty$
  som videre betyr at
  \[\ker\mu_\sigma = \im\sigma = \bbag{I\cap
  J}{(I,J)\in\sigma}.\]
  La $(L,J)\in\kappa_{\mu_\sigma}$ da er
  $L=\ker(\mu_\sigma,J)$. Dermed hvis $\mu_\sigma$ ikke
  parvis kobler $L$ er $L=J$. Hvis $(J,K)\in\mu_\sigma$
  er $K = \coker(\sigma,J)$ som er $J$ hvis $\sigma$ ikke
  parvis kobler $J$, da er $K=\empty$ og bryter
  overlapping, eller så er $(I,J)\in\sigma$ og $K = J-I$.
  I tilfellet der $(I,J)\in\sigma$ er $L=J-K=J-(J-I)$ som
  tilslutt er $I\cap J$. Dermed er $\omega_\sigma(I\cap
  J)=J$, eller sagt annerledes er $\omega_\sigma$
  multimengden av par $(I\cap,J,J)$ hvor $(I,J)\in\sigma$.
}
