\section{Varighetsmoduler}

% Når vi skal studere homologien filtrasjonene av topologiske rom og Čech-/Rips-komplekser er det gunstig å kunne samle alle vektorrommene gitt av homologiene av filtrasjonen av rommet.

% For dette er det naturlig å bruke varighetsmoduler
Et sentralt tema for å kunne forstå stabilitet og topologisk dataanalyse er ideen om varighetsmoduler. I dette kapitellet går vi gjennom en litt abstrakt introduksjon og så ser vi på hvorfor de er viktige innenfor topologisk dataanalyse. Definisjonen på en varighetsmodul er kort og enkel.

\begin{definisjon}\label{Def:PersMod}
    En varighetsmodul $M$ er en funktor $M:\Rbld\to\vect_k$.
\end{definisjon}

Vi skriver $M_t$ for vektorromet $M(t)$ (det $t$-ende vektorromet) for å unngå fremtidig forvirring.
Siden en varighetsmodul $M$ er en funktor fra pomengden $\Rbld$ til $\vect_k$ så har vi for hver $s\leq t$ en lineær avbilding $\p_M(s,t): M_s\to M_t$ som vi kaller overgangsavbildinger.

Gitt to varighetsmoduler $M$ og $N$ kan vi definere en morfi $f: M\to N$ som en samling av lineære avbildinger
$\{f(s): M_s\to N_s\;|\; s\in\Rbb\}$ slik at diagrammet
\[
\begin{tikzcd}
	{M_s} && {N_s} \\
	\\
	{M_t} && {N_t}
	\arrow["{f(s)}", from=1-1, to=1-3]
	\arrow["{\p_M(s,t)}", from=1-1, to=3-1]
	\arrow["{\p_N(s,t)}", from=1-3, to=3-3]
	\arrow["{f(t)}", from=3-1, to=3-3]
\end{tikzcd}
\]
kommuterer. Vi kan komponere morfiene på den følgende måten; gitt morfier $f:M\to N$ og $g:M\to P$ er $g\circ f$
definert som samlingen $\{g_s\circ f_s: M_s\to P_s\;|\;s\in\Rbb\}$.

Siden vi har objekter, varighetsmoduler, og vi har morfier mellom dem kan vi definere kategorien av varighets moduler

\begin{definisjon}\label{Def:KatPMod}
    Kategorien $\vect_k^\Rbld$ er kategorien av varighetsmodulene med varighetsmodul-morfier mellom dem.
\end{definisjon}

\subsection{Inflettingsavstand}
Stabilitet av varighetsmoduler innebærer relasjonen mellom to typer distanser, Bottleneck distansen mellom barkoder
og inflettingsavstanden mellom varighetsmoduler. Her definerer vi inflettingsavstanden mellom to varighetsmoduler.

For å definere distansen må vi gjennom noen få steg.
\begin{definisjon}\label{Def:DShift}
	En $\delta$-forskyvning av en varighetsmodul er en funktor
	\[(\cdot)(\delta): \Pers\to \Pers\]
	Som tar en varighetsmodul $M$ til $M(\delta)$ hvor $M(\delta)_t = M_{t+\delta}$ og tar varighetsmodulmorfier $f:M\to N$ til $f(\delta):M(\delta)\to N(\delta)$.
\end{definisjon}

Denne funktorer gir oss konseptet av $\delta$-infletting.

\begin{definisjon}\label{Def:infletting}
	La $M$ og $N$ være varighetsmoduler. Vi sier at $M$ og $N$ er $\delta$-interleavet hvis det eksisterer varighetsmodulmorfier $f:M\to N(\delta)$ og $g:N\to M(\delta)$ slik at
	\[g(\delta)\circ f = \p_M(t,t+2\delta),\quad f(\delta)\circ g = \p_N(t,t+2\delta)\]
\end{definisjon}
Vi kaller $\p^\eps_M(t) = \p_M(t,t+\eps)$. Bemerk at $\p^0_M=\id_M$ fordi $\p^0_M(t) = \p_M(t,t+0)=\p_M(t,t)=\id_M$.

\begin{definisjon}\label{def:UtPsMet}
   En utvidet pseudometrikk på en klasse $X$ er en
   funksjon $d: X\times X\to [0,\infty]$ med følgende
   egenskaper:
   \begin{itemize}
     \item $d(x,x) = 0$ for enhver $x\in X$,
     \item $d(x,y) = d(y,x)$ for enhver $x,y\in X$
     \item $d(x,z) \leq d(x,y) + d(y,x)$ for enhver
       $x,y,z\in X$ slik at $d(x,y),d(y,z)\leq\infty$.
   \end{itemize}
\end{definisjon}

Forskjellen på en utvidet pseudometrikk og en metrikk er
at det kan være to elementer $x,y\in X$ slik at $x\neq y$,
men $d(x,y)=0$ i motsetning til metrikker der $d(x,y)=0$
hvis og bare hvis $x=y$.

\begin{definisjon}\label{Def:InfAvs}
	For $M$ og $N$ varighetsmoduler definerer vi inflettingsavstanden $d_I$ ved
	\[d_I(M,N) = \inf\{\delta\in \ropen{0}{\infty}\mid \text{$M$ og $N$ er $\delta$-interleavet}\}\]
\end{definisjon}
Denne avstanden gir et tall på hvor "isomorfe" to varighetsmoduler er.

\begin{teorem}\label{trm:Utvidet-Metrikk}
  Inflettingsavstanden er en utvidet metrikk
\end{teorem}
\begin{proof}
    
\end{proof}

\begin{proposisjon}\label{prop:d_I-Iso}
	For $M$ og $N$ varighetsmoduler så holder
	\[d_I(M,N) = 0 \iff M\cong N\]
\end{proposisjon}
\begin{proof}
	"$\implies$"\\\\
	Hvis $d_I(M,N) = 0$ så finnes det en $0$-infletting mellom $M$ og $N$ altså det eksisterer varighetsmodulmorfier $f: M \to N(0)=N$ og $g: N\to M(0)=M$ slik at $g(0)\circ f = g\circ f = \p_M^0=\id_M$ og $f(0)\circ g = \p_N^0=\id_N$. Dermed er $f$ og $g$ inverser av hverandre og er dermed isomorfier.\\\\
	"$\impliedby$"\\\\
	Hvis $M\cong N$ så eksisterer det varighetsmodulmorfier $f:M\to N$ og $g:N\to M$ slik at $g\circ f = \id_M=\p_M^0$ og $f\circ g=\id_N=\p_N^0$. Så det eksisterer en $0$-infletting og dermed er $d_I(M,N) = 0$.
\end{proof}


\begin{proposisjon}\label{prop:InfAvs_Metrikk}
  Hvis $M\cong M'$ er isomorfe varighetsmoduler så er
  $d_I(M,N) = d_I(M',N)$ for enhver varighetsmodul.
\end{proposisjon}\begin{proof}
  Siden $d_I$ er en utvidet metrikk betyr det at den
  oppfyller den trekantulikheten $d_I(M,N)\leq
  d_I(M,M')+d_I(M',N)$. Ved
  \bref{proposisjon}{prop:d_I-Iso} er $d_I(M,M')=0$ altså
  er $d_I(M,N)\leq d_I(M',N)$. Ved samme argument er
  $d_I(M',N)\leq d_I(M,N)$. Dermed er
  $d_I(M,N)=d_I(M',N)$.
\end{proof}

Resultatet fra \bref{proposisjon}{prop:InfAvs_Metrikk}
betyr at inflettingsavstanden er en metrikk på
isomorfiklassene av varighetsmodulene.


