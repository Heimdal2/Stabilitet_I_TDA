\section{Varighetshomologi}\label{sek:VarHom}
I denne seksjonen går vi over hvordan man gjør data fra et
eksperiment eller en statistisk undersøkelse om til topologiske
rom og videre om til varighetsmoduler. Vi starter med å lage
varighetsmoduler av generelle topologiske rom, deretter går vi
over forskjellige måter å lage topologiske rom fra data i form av
punktskyer.

\begin{definisjon}\label{def:FiltTop}
En filtrering av et topologisk rom $X$ over en pomengde $(P,\leq)$
er en samling $\{X_p\}_{p\in P}$ slik at når $q\leq p$ så er
  $X_q\subset X_p$ og $\bigcup_{p\in P} X_p = X$.
\end{definisjon}

En kan også tenke på en filtrering av et rom $X$ over en pomengde
$\Pbld$ som en funktor $X_\bullet = \Pbld\to\Top$ slik at morfien
$q\leq p$ i $\Pbld$ blir sendt til inklusjonen $i_X(q,p):
X_q\monicto X_p$.

I denne oppgaven er en filtrering av et rom alltid over $\Rbld$
pomengden.

Gitt et topologisk rom, kan vi kalkulere homologigruppene dens.
Vi bruker koeffisienter fra en kropp $K$ til å kalkulere
homologigruppene, da får vi vektorrom.

Vi kan lage en varighetsmodul av et rom ved å filtrere rommet og
så ta homologien av rommene i filtreringen.

\begin{definisjon}\label{def:VarHom}
  La $X$ være et topologisk rom med filtrering $X_\bullet$ en
  filtrering av $X$. Da er den $n$-te varighetshomologien
  $\Hcal_n(X_\bullet):
  \Rbld\to\vect_K$ definert ved $\Hcal_n(X_\bullet)_s = H_n(X_s)$ og
  $\p_{\Hcal_n(X_\bullet)}(s,t) = i_X(s,t)_\ast$.
\end{definisjon}

Nå som vi har definisjonen på varighetshomologien av et rom går vi
videre til å lage topologiske rom av data i form av en filtrering
av et rom.

I \bref{Seksjoner}{sek:Forkunnskaper} gikk vi over
$\Delta$-komplekser. Rommene vi lager fra punktskydataen er slike
$\Delta$-komplekser. Det er to hovedmetoder for å lage rom av
punktskyer, kalt Čech-komplekser og Vietoris-Rips-komplekser også
kalt Rips-komplekser.

Følgende definisjoner er inspirert fra \cite{Schenck2022}

\begin{definisjon}\label{def:CechKomp}
  La $P\subset\Rbb^d$ være en punktsky og la $B_\eps(p)$ være den
  lukkede $\eps$-ballen om et punkt $p\in P$. Čech-komplekset
  $\Cpzc_\eps$ av $P$ inneholder $k$-simplekser som korresponderer
  med $k+1$-tupler av punkter $p\in P$ slik at
  \[\bigcap_{i=0}^k \overline{B_\frac{\eps}{2}(p_i)}\neq\emptyset.\]
  Altså er en $k+1$-tuppel $(p_0,\dots,p_k)\in\Cpzc$ hvis deres
  lukkede $\eps$-baller deler minst et punkt.
\end{definisjon}

\begin{definisjon}\label{def:RipsKomp}
  La $P\subset\Rbb^d$ være en punktsky. Rips-komplekset
  $\Rpzc_\eps$ av $P$ gitt ved $k$-simplekser som korresponderer
  med $k+1$-tupler $(p_0,\dots,p_k)$ slik at $|p_i-p_j|\leq\eps$
  for alle par $i,j$.
\end{definisjon}


