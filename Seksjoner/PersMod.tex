\section{Persistensmoduler og Barkoder}

\subsection{Persistensmoduler}
% Når vi skal studere homologien filtrasjonene av topologiske rom og Čech-/Rips-komplekser er det gunstig å kunne samle alle vektorrommene gitt av homologiene av filtrasjonen av rommet.

% For dette er det naturlig å bruke Persistensmoduler
Et sentralt tema for å kunne forstå stabilitet og topologisk dataanalyse er ideen om persistensmoduler. I dette kapitellet går vi gjennom en litt abstrakt introduksjon og så ser vi på hvorfor de er viktige innenfor topologisk dataanalyse. Definisjonen på en Persistensmodul er kort og enkel.

\begin{definition}\label{Def:PersMod}
    En persistensmodul $M$ er en funktor $M:\Rbold\to\vect_k$.
\end{definition}

Vi skriver $M_t$ for vektorromet $M(t)$ (det $t$-ende vektorromet) for å unngå fremtidig forvirring.
Siden en persistensmodul $M$ er en funktor fra pomengden $\Rbold$ til $\vect_k$ så har vi for hver $s\leq t$ en lineær avbilding $\p_M(s,t): M_s\to M_t$ som vi kaller overgangsavbildinger.

Gitt to persistensmoduler $M$ og $N$ kan vi definere en morfi $f: M\to N$ som en samling av lineære avbildinger $\{f(s): M_s\to N_s\;|\; s\in\R\}$ slik at diagrammet
\[
\begin{tikzcd}
	{M_s} && {N_s} \\
	\\
	{M_t} && {N_t}
	\arrow["{f(s)}", from=1-1, to=1-3]
	\arrow["{\p_M(s,t)}", from=1-1, to=3-1]
	\arrow["{\p_N(s,t)}", from=1-3, to=3-3]
	\arrow["{f(t)}", from=3-1, to=3-3]
\end{tikzcd}
\]
kommuterer. Vi kan komponere morfiene på den følgende måten; gitt morfier $f:M\to N$ og $g:M\to P$ er $g\circ f$ definert som samlingen $\{g_s\circ f_s: M_s\to P_s\;|\;s\in\R\}$.

Siden vi har objekter, persistensmoduler, og vi har morfier mellom dem kan vi definere kategorien av persistens moduler

\begin{definition}\label{Def:KatPMod}
    Kategorien $\vect_k^\Rbold$ er kategorien av persistensmodulene med persistensmodul-morfier mellom dem.
\end{definition}

\begin{example}\label{Ex:HomologiEks}
	Gitt en filtrering $F_\bullet X = \{F_tX\}_{t\in\R}$ av et topologisk rom $X$ kan vi definere persistensmodulen $H_n(F_\bullet X) = \{H_n(F_tX;k)\}_{t\in\R}$ med overgangsavbildinger $\p_{H_n(F_\bullet X)}(s,t) = i_*(s,t)$
\end{example}

Et eksempel på en særlig enkel, men viktig persistensmodul er interval-persistensmodulen definert som følgende.

La $I\subset\R$ være et intervall da definerer vi intervall-persistensmodulen $C(I)$ som følger:

\[C(I)_t = 
\begin{cases}
    k,\quad t\in I\\
0,\quad\text{ellers}
\end{cases}
\]

med overgangsavbildinger definert ved

\[\p_{C(I)}(s,t) = 
\begin{cases}
id_k,\quad s,t\in I\\
0, \quad\text{ellers}
\end{cases}
\]
Denne persistensmodulen er nyttig når vi skal definere barkoder snart.

\subsubsection{Interleaving-distanse}
Stabilitet av persistensmoduler innebærer relasjonen mellom to typer distanser, Bottleneck distansen mellom barkoder og Interleaving distansen mellom persistensmoduler. Her definerer vi interleaving distansen mellom to Persistensmoduler.

For å definere distansen må vi gjennom noen få steg.
\begin{definition}\label{Def:DShift}
	En $\delta$-forskyvning av en persistensmodul er en funktor
	\[(\cdot)(\delta): \Pers\to \Pers\]
	Som tar en persistensmodul $M$ til $M(\delta)$ hvor $M(\delta)_t = M_{t+\delta}$ og tar persistensmodulmorfier $f:M\to N$ til $f(\delta):M(\delta)\to N(\delta)$.
\end{definition}

Denne funktorer gir oss konseptet av $\delta$-interleavinger.

\begin{definition}\label{Def:interleaving}
	La $M$ og $N$ være persistensmoduler. Vi sier at $M$ og $N$ er $\delta$-interleavet hvis det eksisterer persistensmodulmorfier $f:M\to N(\delta)$ og $g:N\to M(\delta)$ slik at
	\[g(\delta)\circ f = \p_M(t,t+2\delta),\quad f(\delta)\circ g = \p_N(t,t+2\delta)\]
\end{definition}
Vi kaller $\p^\eps_M(t) = \p_M(t,t+\eps)$. Bemerk at $\p^0_M=\id_M$ fordi $\p^0_M(t) = \p_M(t,t+0)=\p_M(t,t)=\id_M$.

\begin{definition}\label{Def:label}
	For $M$ og $N$ persistensmoduler definerer vi interleaving-distansen $d_I$ ved
	\[d_I(M,N) = \inf\{\delta\in \ropen{0}{\infty}\mid \text{$M$ og $N$ er $\delta$-interleavet}\}\]
\end{definition}
Denne avstanden gir et tall på hvor "isomorfe" to persistensmoduler er.

\begin{proposition}\label{Cor:label}
	For $M$ og $N$ persistensmoduler så holder
	\[d_I(M,N) = 0 \iff M\cong N\]
\end{proposition}
\begin{proof}
	"$\implies$"\\\\
	Hvis $d_I(M,N) = 0$ så finnes det en $0$-interleaving mellom $M$ og $N$ altså det eksisterer persistensmodulmorfier $f: M \to N(0)=N$ og $g: N\to M(0)=M$ slik at $g(0)\circ f = g\circ f = \p_M^0=\id_M$ og $f(0)\circ g = \p_N^0=\id_N$. Dermed er $f$ og $g$ inverser av hverandre og er dermed isomorfier.\\\\
	"$\impliedby$"\\\\
	Hvis $M\cong N$ så eksisterer det persistensmodulmorfier $f:M\to N$ og $g:N\to M$ slik at $g\circ f = \id_M=\p_M^0$ og $f\circ g=\id_N=\p_N^0$. Så det eksisterer en $0$-interleaving og dermed er $d_I(M,N) = 0$.
\end{proof}

\subsection{Multimengder}
Mengder er begrenset i og med at de ikke inneholder repitisjoner, mengden $\{a,a,b\}$ er regnet som mengden $\{a,b\}$. For oss vil vi ha muligheten for at en mengde kan inneholde mange like elementer.

Dermed definerer vi en multimengde.
\begin{definition}\label{Def:label}
    Vi definerer en multimengde som et par $\Scal = (S,m)$, hvor $S$ er en mengde og en funksjon $m: S\to\N$.
\end{definition}

Multimengder er derimot vanskelige å jobbe med, derfor jobber vi med deres representasjoner
\[\Rep(\Scal) = \{(s,k)\in S\times\N\;|\; k \leq m(s)\}.\]

\subsection{Barkoder}
En barkode $\Bcal$ er en representasjon av en multimengde av intervaller. Elementer i en barkode er dermed par $(I,k)$ der $I$ er et intervall og $k\in\N$. Ofte når indeksen $k$ er nødvendig skriver vi bare $I$ for et intervall i barkoden.

I \cite{Bauer2015a} sier forfatter at gitt en persistensmodul $M$ som kan skrives
\[M\cong\bigoplus_{I\in\Bcal_M}C(I)\]
Da er $\Bcal_M$ unikt bestemt. Vi kaller slike persistensmoduler intervalldekomponerbare.

Dette er en konsekvens av følgende teorem

\begin{theorem}\label{Thrm:label}
    Hvis $\bigoplus_{I\in\Bcal}C(I)\cong\bigoplus_{J\in\Ccal} C(J)$ så er $\Bcal\cong\Ccal$
\end{theorem}
\begin{proof}
Anta at det ikke eksisterer en bijeksjon mellom $\Bcal$ og $\Ccal$ f.eks. 
\end{proof}

I følge \cite{Bauer2015a} har vi følgende teorem
\begin{theorem}\label{Thrm:thrm2.1}
	Enhver p.e.d. persistensmodul er intervalldekomponerbar.
\end{theorem}
\begin{proof}
	La $M$ være en p.e.d. peristensmodul. Mengden $I_n = \{s\in\R\;\mid\;\dim M_s=n\}$ er en disjunkt union av intervaller $I_{n_1}\sqcup I_{n_2}\sqcup\dots$, vi kan la $C(I_{n_1}\sqcup I_{n_2}\sqcup\dots)=C(I_{n_1})\oplus C(I_{n_2})\oplus\dots$ og få
	\[M\cong \bigoplus_{n,k\in\in\N} C(I_{n_k}).\]
	Da blir $\Bcal_M = \{I_{n_j}\;\mid\; n,j\in\N\}$, en kan la $I_{n_j}=\es$ når den ikke påvirker unionen $I_n$. 
\end{proof}

Liknende for persistensmoduler har barkoder sin egen metrikk, men for å komme fram til denne metrikken må en gå gjennom noen definisjoner

\subsection{Matching-kategorien}
Definisjonen på en matching er gitt i \cite{Bauer2015a} seksjon 2.2
\begin{definition}\label{Def:Matching}
	En matching mellom mengder $S$ og $T$ (skrevet $\sigma:S\pto T$) er en bijeksjon $\sigma: S'\to T'$ mellom delmengder $S'\subset S$ og $T'\subset T$. Vi skriver $T'$ som $\im\sigma$ og $S' = \coim\sigma$.
\end{definition}
Fra \cite{Bauer2015a} har vi også følgende definisjon: "En kan tenke på en matching $\sigma$ som en relasjon $\sigma\subset S\times T$ slik at $(s,t)\in \sigma$ hvis og bare hvis $s\in\coim\sigma$ og $\sigma(s)=t$."
\cite{Bauer2015a} nevner også revers-matchingen $\rev\sigma$, men skriver bare at den er definert på den åpenbare måten. Her er en rask og litt grundigere definisjon på $\rev$ siden den vil bli brukt senere.

\begin{definition}\label{Def:rev}
	For en matching $\sigma: S\pto T$ er et element $(t,s)\in\rev\sigma$ hvis og bare hvis $(s,t)\in\sigma$.
\end{definition}

Vi kan også komponere matchinger som definert i \cite{Bauer2015a}. La $\sigma: S\pto T$ og $\tau: T\pto U$ være matchinger da kan vi definere komposisjonen

\[\tau\circ\sigma = \{(s,u)\;\mid\; \exists t\in T\sa (s,t)\in\sigma, (t,u)\in\tau\}\]

Dette gjør matchinger av mengder om til en kategori $\Mch$, hvor objektene er mengder og morfiene er matchinger.

\subsection{Dekorerte endepunkter}
Matchingene vi vil undersøke er matchinger mellom barkoder som beveger endepunkter til intervaller på en kontollert måte. Dette gjøres i den $\Rbold$-indiserte innstillingen ved å introdusere litt formalisme.

La $D = \{-,+\}$ være en totalt ordnet mengde med $-\leq+$. La $\E = \R\times
D\cup\{-\infty,\infty\}$ være mengden av dekorerte endepunkter. For en $t\in\R$
skriver vi $t^+$ og $t^-$ istedet for å skrive $(t,+)$ og $(t,-)$ hhv. Den
leksikografiske ordenen definert i \bref{eksempel}{Ex:LeksOrd} induserer en total orden på
$\E$ definert ved

\[s^\pm\leq t^\pm\\leq t^\pm\]
