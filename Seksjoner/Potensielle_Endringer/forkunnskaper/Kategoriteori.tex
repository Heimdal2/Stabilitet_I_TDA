\section{Kategoriteori}

Følgende er et sitat fra \textbf{"Ref til Mac Lane Her"} "Category
theory starts with the observation that many properties of
mathematical systems can be unified and simplified by
a presentation with diagrams and arrows." Dette rammeverket kommer
til å være veldig nyttig i denne oppgaven, men for å bruke det må
vi vite bedre hva som menes med piler og diagrammer.

\begin{definisjon}\label{def:Kat}
En kategori $\Ccal$ er en tuppel
$(\Ob(\Ccal),\hom(\Ccal),\dom,\cod,\circ)$ med følgende
egenskaper:
\begin{itemize}
\item $\Ob(\Ccal)$ er en klasse av elementer vi kaller objekter
\item $\hom(\Ccal)$ er en klasse av piler, kalt morfier, $f:a\to b$ for et par av objekter $a$ og $b$.
\item $\dom,\cod:\hom(\Ccal)\to\Ob(\Ccal)$ er funksjoner slik at
for en morfi $f:a\to b$, så er $\dom(f) = a$ og $\cod(f) = b$. Vi
kaller $a$ domenen til $a$ og $b$ codomenen.
\item For ethvert objekt $A$ i $\Ccal$ eksister det en morfi
$\id_a:a\to a$, med egenskapen at for enhver morfi $f:a\to b$ så
er $f\circ\id_a = f$ og forenhver $g: b\to a$ så er $\id_a\circ g=
g$.
\item $\circ$ er en binæroperasjon kalt en komposisjon som tar et par av morfier på
formen $f:a\to b$ og $g: b\to c$ og gir en ny morfi $g\circ f:
a\to c$. For å komponere en morfi $g$ med en morfi $f$ trenger vi
at domenen til $g$ er lik kodomenen til $f$.
\end{itemize}
\end{definisjon}

Nå ser vi på noen eksempler på noen vanlige kategorier.

\begin{eksempel}\label{eks:Setkat}
  Hvis vi lar $\Ob(\Ccal)$ være klassen av alle mengder og
  $\hom(\Ccal)$ være de vanlige funksjonene mellom mengder, får vi
  kategorien av mengder ofte skrevet $\Set$. Gitt to funksjoner
  $f:A\to B$ og $g:B\to C$ er komposisjon definert punktvis ved
  $(g\circ f)(x) = g(f(x))$. For en mengde $A$ er
  identitetsmorfien $\id_A$ gitt ved å sende et element i $A$ til
  seg selv.
\end{eksempel}

\begin{eksempel}\label{eks:Topkat}
  Vi kan også la $\Ob(\Ccal)$ være klassen av topologiske rom med
  morfier kontinuerlige funksjoner. Komposisjon er likt definert
  som i $\Set$ og komposisjon av kontinuerlige funksjoner er også
  kontinuerlig. Igjen likt som i $\Set$ er identitetsmorfien for
  et topologisk rom $\id_X$ definert ved å sende et punkt til
  seg selv. Kategorien av topologiske rom er skrevet $\Top$.
\end{eksempel}

\begin{eksempel}\label{eks:Vektkat}
  En annen kategori er hvor objektene er vektorrom og morfiene er
  lineære avbildinger mellom rommene. Komposisjon og
  identitetsmorfiene er definert likt som i $\Set$ og $\Top$. Vi
  kaller kategorien av vektorrom for $\Vect$ og vektorrom over en
  viss kropp $\Vect_K$.
\end{eksempel}

Det er også viktig å observere at det er ingen sted i definisjonen
til en kategori at objektene trenger å ha en underliggende
mengdestruktur og at morfiene trenger å være funksjoner. Her er et
eksempel på en slik kategori.

\begin{eksempel}\label{eks:Posetkat}
  La $P$ være en delvis ordnet mengde med orden $\leq$. Vi kan la
  $\Ccal$ være en kategorien definert ved å la objektene vær
  elementene i $P$ og vi har en morfi $a\to b$ hvis $a\leq b$ som
  elementer i $P$. Komposisjon blir da gitt ved transitivitet,
  hvis vi har morfier $f:a\to b$ og $g:b\to c$, da har vi at
  $a\leq b$ og $b\leq c$ og ved transitivitet vet vi at $a\leq c$
  som betyr at det er en morfi $a\to c$ som vi kaller $g\circ f$.
\end{eksempel}

Det er også viktig å notere at for enhver kategori $\Ccal$ har vi
den motsatte kategorien skrevet $\Ccal^\op$.

\begin{remark}\label{rem:smaakat}
Objektene og morfiene til noen kategorier som f.eks. $\Set$ som
vi senere snakker raskt om kan ikke være inneholdt i det vi
kaller mengder og er heller elementer av en klasse. Vi kaller
kategorier hvor objektene og morfiene er elementer av en mengde
små kategorier eller en liten kategori. Disse kategoriene kommer til å bli brukt senere i definisjonen på et diagram.
\end{remark}

%DETTE ER TATT FRA
%https://ncatlab.org/nlab/show/opposite+category MÅ ENDRES ELLER
%REFERERES TIL.
\begin{definisjon}\label{def:OpKat}
  Gitt en kategori $\Ccal$ eksisterer det en kategori $\Ccal^\op$
  hvor alle objektene $\Ccal$ er lik i $\Ccal^\op$, men alle
  morfiene bytter retning. En morfi $f: A\to B$ i $\Ccal^\op$
  lik en morfi $f: B\to A$ i $\Ccal^\op$. For morfier $f: A\to B$
  og $g: B\to C$ er komposisjonen $g\circ f$ i $\Ccal^\op$
  definert som komposisijonen $f\circ g$ i $\Ccal$. 
\end{definisjon}

Det eksisterer også piler mellom kategorier kalt funktorer.

\begin{definisjon}\label{def:Funktor}
  Gitt kategorier $\Ccal$ og $\Dcal$ er en funktor
  $F:\Ccal\to\Dcal$ definert på følgende måte.
  \begin{itemize}
    \item For et objekt $A$ i $\Ccal$ er $F(A)$ et objekt
      i $\Dcal$.
    \item For en morfi $f:A\to B$ i $\Ccal$ er $F(f): F(A)\to
      F(B)$ en morfi i $\Dcal$.
    \item For et objekt $A$ i $\Ccal$ er $F(\id_A)=\id_{F(A)}$.
    \item For morfier $f: A\to B$ og $g: B\to C$ er $F(g\circ f)
      = F(g)\circ F(f)$.
  \end{itemize}
\end{definisjon}

\begin{eksempel}\label{eks:GlemmeFunktor}
  Funktoren $F:\Top\to\Set$ som sender et topologisk rom
  $(X,\Tcal)$ til sin underliggende mengde $X$ og kontinuerlige
  funksjoner $f:(X,\Tcal)\to(Y,\Tcal')$ blir sendt til dens
  underliggende mengdefunksjon. Siden identitetsmorfiene i $\Top$
  er definert likt som identitetsmorfiene i $\Set$ så er
  $F(\id_{(X,\Tcal)})=\id_{F((X,\Tcal))}=\id_X$. Siden komposisjon
  i $\Top$ er lik som i $\Set$ har vi at $F(g\circ f) = F(g)\circ 
  F(f)$. Dette betyr at $F:\Top\to\Set$ er en funktor.
\end{eksempel}

En spesiell type funktor er følgende.
\begin{definisjon}\label{def:KontraFunktor}
En funktor $F:\Ccal^\op\to\Dcal$ eller $F:\Ccal\to\Dcal^\op$ sies
å være kontravariant og vi skriver ofte bare $F:\Ccal\to\Dcal$ med
tillegsinformasjonen om at den er kontravariant. Kontravariante
funktorer har egenskapen at de bytter retning på morfier, hvis $F$
er kontravariant og $f: A\to B$ er en morfi, da er $F(f): F(B)\to
F(A)$.
\end{definisjon}




\begin{definisjon}\label{def:Diagram}
  
\end{definisjon}
