\subsection{Kategoriteori}\label{Ssek:Katgoriteori}
%Et samlende rammeverk i matematikk er en vanskelig oppgave å få laget, men et rammeverk som gjør en god jobb og som brukes heletiden nå til dags er kategoriteori. Kategoriteori gir en formulering på de forskjellige områdene i matematikk som f.eks. topologi og algebra.

Forfatter i \citep[introduksjon]{Lane2010} sier "Category
theory starts with the observation that many properties of
mathematical systems can be unified and simplified by
a presentation with diagrams of arrows." Kategoriteori lar
oss samle forskjellige områder innen matematikk som
topologiske rom og vektorrom. Teorien lar oss også studere
egenskaper i en kategori ved bruk av en annen kategori via
spesielle piler kalt funktorer.

Fra \citep[definisjon 1.2.1]{Agore2023} er definisjonen på en kategori
følgende

\begin{definisjon}\label{def:Kategori}
  En kategori $\Ccal$ består av den følgende data:
  \begin{enumerate}
    \item En klasse $\Ob(\Ccal)$ med elementer kalt
      objekter.
    \item for ethvert par av objekter $A$, $B$ en (mulig
      rom) mengde $\Hom_\Ccal(A,B)$ med elementer kalt
      morfier fra $A$ til $B$. Et element
      $f\in\Hom_\Ccal(A,B)$ denoteres $f: A\to B$, $A$ og
      $B$ er kalt domenen og kodomenen av $f$ hhv.
    \item for enhver trippel av objekter $A,B,C$, en
      komposisjonslov
      \[\Hom_\Ccal(A,B)\times\Hom_\Ccal(B,C)\to\Hom_\Ccal(A,C)\]
      \[(f,g)\mapsto g\circ f\]
    \item for ethvert objekt $A$ en morfi
      $\id_A\in\Hom_\Ccal(A,A)$ kalt identiteten på $A$.
    \item Assosiativ lov: for morfier
      $f\in\Hom_\Ccal(A,B)$, $g\in\Hom_\Ccal(B,C)$ og
      $h\in\Hom_\Ccal(C,D)$ holder følgende likning
      \[h\circ(g\circ f) = (h\circ g)\circ f\]
    \item Identitetslov: for en morfi
      $f\in\Hom_\Ccal(A,B)$ holder likningen
      \[\id_B\circ f = f = f\circ \id_A.\]
  \end{enumerate}
  En kategori hvor klassen av objekter er en mengde er
  kalt en liten kategori og en kategori med endelig mange
  morfier er kalt endelig.
\end{definisjon}

%\begin{definisjon}\label{Def:Kategori}
%    En kategori $\Ccal$ er et par $(\Ob(\Ccal),\hom(\Ccal))$ av klasser. Elementene i $\Ob(\Ccal)$ er kalt objekter og elementene i $\hom(\Ccal)$ er kalt morfier. Morfiene i $\Ccal$ kan bli sett på som piler mellom objektene i $\Ccal$, en morfi $f$ mellom objekter $A$ og $B$ skrives $f:A\to B$.
%    For morfier $f: A\to B$ og $g: B\to C$ har vi en morfi $g\circ f: A\to C$ som vi kaller komposisjonen av $f$ med $g$.
%    For ethvert objekt $A\in\Ob(\Ccal)$ har vi en morfi $\id_A: A\to A$ som vi kaller identitetsmorfien som tilfredstiller for enhver morfi $f:A\to B$:
%    \[f\circ\id_A = f,\quad \id_B\circ f = f.\]
%\end{definisjon}

%\begin{bemerk}\label{rem:smaakat}
%  Objektene og morfiene til noen kategorier som f.eks. $\Set$ som
%  vi senere snakker raskt om kan ikke være inneholdt i det vi
%  kaller mengder og er heller elementer av en klasse. Vi kaller
%  kategorier hvor objektene og morfiene er elementer av en mengde
%  små kategorier eller at en kategori er liten. Disse kategoriene kommer til å bli brukt senere i definisjonen på et diagram.
%\end{bemerk}

I en kategori er det noen spesielle morfier som kalles
isomorfier, monomorfier og epimorfier.
Fra \citep[definisjon 1.3.1]{Agore2023} har vi følgende
definisjoner for disse morfiene.
\begin{definisjon}\label{Def:Iso}
    En isomorfi i en kategori $\Ccal$ er en morfi $f:A\to B$ mellom to objekter $A,B\in\Ob(\Ccal)$ hvor det eksisterer en morfi $g:B\to A$ slik at følgende holder
    \[f\circ g = \id_B,\quad g\circ f = \id_A.\]
    Vi kaller $g$ inversen til $f$. Hvis to objekter $A$ og $B$ er ismorfe skriver vi $A\cong B$.
\end{definisjon}

\begin{bemerk}\label{Rem:IdIso}
    Identitetsmorfien er en isomorfi siden $\id_A\circ\id_A = \id_A$. Altså er $\id_A$ sin egen invers.
\end{bemerk}

Andre spesielle morfier er følgende:
\begin{definisjon}\label{Def:Epi}
    En morfi $f: A\to B$ er kalt en epimorfi hvis for
    ethvert par med mofier $g,h: B\to C$ så holder
    \[g\circ f = h\circ f \implies g = h\]
    En skriver $f:A\epito B$ for en epimorfi når det er
    viktig å notere.
\end{definisjon}

\begin{definisjon}\label{Def:Mono}
    En morfi $f: A\to B$ er kalt en monomorfi hvis for ethvert par $g,h: C\to A$ så holder
    \[f\circ g = f\circ h \implies g=h\]
    En skriver $f: A\monicto B$ for en monomorfi hvis det er viktig å notere.
\end{definisjon}

Noen eksempler på kategorier, deres isomorfier, epimorfier og monomorfier er følgende
\begin{eksempel}\label{Ex:Set}
    Kategorien $\Set$ har mengder som objekter og
    funksjoner som morfier. Her er bijektive funksjoner
    isomorfier, surjektive funksjoner epimorfier og
    injektive funksjoner er monomorfier.
\end{eksempel}

\begin{eksempel}\label{Ex:TopKat}
    Kategorien $\Top$ er kategorien hvor objektene er topologiske rom og morfiene er kontinuerlige funksjoner. Isomorfier i $\Top$, kalt homeomorfier, er bijektive og kontinuerlig funksjoner med en kontinuerlig invers, epimorfi er surjektive og kontinuerlige funksjoner og monomorfier er injektive og kontinuerlige funksjoner. 
\end{eksempel}

\begin{eksempel}\label{Ex:VektKat}
    Kategorien $\Vect_K$ er kategorien av vektorrom over en kropp $K$ som objekter og lineære avbildinger som morfier. Isomorfier i $\Vect_K$, kalt vektorromisomorfier, er bijektive og lineære avbildinger, inversen vil automatisk være lineær så vi trenger ikke inverskriteriet som vi gjør i $\Top$. Epimorfiene er surjektive lineære avbildinger og monomorfiene er injektive lineære avbildinger.
\end{eksempel}

Morfiene i en kategori trenger ikke å være funksjoner, her er et eksempel på en kategori hvor morfiene ikke er funksjoner.
\begin{eksempel}\label{Ex:RPoset}
    Kategorien $\Rbld$ har de reelle tall $\Rbb$ som objekter og $\leq$ relasjonen som morfier. Komposisjon er gitt ved transitivitet og identitetsmorfiene er gitt ved $s=s$. Identitetsmorfiene er også de eneste isomorfiene fordi hvis $a\leq b$ er en isomorfi så er $b\leq a$ dens invers, men hvis $a\leq b$ og $b\leq a$ så er $a=b$. Her er alle morfier epimorfier og monomorfier.
\end{eksempel}

For enhver kategori kan vi alltid lage dens
motsatte eller dualkategorien

\begin{definisjon}\label{def:OpKat}
   La $\Ccal$ være en kategori. Dualkategorien
   $\Ccal^\op$ av $\Ccal$ er definert som følger
  \begin{enumerate}
    \item $\Ob(\Ccal^\op) = \Ob(\Ccal)$
    \item $\Hom_{\Ccal^\op}(A,B) = \Hom_\Ccal(B,A)$. Vi
      skriver $f^\op: A\to B$ for morfien i $\Ccal^\op$
      som korresponderer med morfien $f: B\to A$
      i $\Ccal$.
    \item Komposisjonsavbildingen
      \[\circ^\op:
      \Hom_{\Ccal^\op}(A,B) \times \Hom_{\Ccal^\op}(B,C)
      \to \Hom_{\Ccal^\op}(A,C)\]
      er gitt ved
      \[g^\op\circ^\op f^\op = (f\circ
      g)^\op\quad\text{for alle
      $f^\op\in\Hom_{\Ccal^\op}(A,B),\,
      g^\op\in\Hom_{\Ccal^\op}(B,C)$.}\]
    \item Identiteter er gitt ved $\id_A^\op = \id_A$ for
      alle objekter.
  \end{enumerate}
\end{definisjon}

%\begin{bemerk}\label{rem:OpKat}
%  Gitt en kategori $\Ccal$ eksisterer det en kategori $\Ccal^\op$
%  kalt den omvendte kategorien av $\Ccal$ med de samme objektene,
%  men med alle morfiene snudd andre vei.
%\end{bemerk}

I noen kategorier er det spesielle objekter kalt initialobjekt og
terminalobjekter. Her er definisjonene deres basert på
\citep[definisjon 1.3.7]{Agore2023}.

\begin{definisjon}\label{def:InitOb}
  I en kategori $\Ccal$ er et initialobjekt $I$ et objekt
  slik at for ethvert objekt $A$ i $\Ccal$ eksisterer det nøyaktig en
  morfi $I\to A$.
\end{definisjon}

\begin{definisjon}\label{def:TermOb}
  I en kategori $\Ccal$ er et terminalobjekt $T$ et
  objekt slik at for ethvert objekt $A$ i $\Ccal$ eksisterer det nøyaktig en morfi $T\to A$.
\end{definisjon}

Et objekt som er både initial og terminal er kalt et nullobjekt.

Vi kan også definere piler mellom kategorier. Disse pilene
er kalt funktorer og er definert som følgende.
I \citep[definisjon 1.5.1]{Agore2023} er en funktor gitt
ved

\begin{definisjon}\label{def:Funktor}
  La $\Ccal$ og $\Dcal$ være to kategorier. En kovariant
  funktor (hhv. kontravariant funktor) $F:\Ccal\to\Dcal$
  består av følgende data:
  \begin{enumerate}
    \item En avbilding $A\mapsto
      F(A):\Ob(\Ccal)\to\Ob(\Dcal)$
    \item for ethvert par av objekter $A,B\in\Ob(\Ccal)$
      en avbilding
      \[f\mapsto
      F(f):\Hom_\Ccal(A,B)\to\Hom_\Dcal(F(A),F(B))\]
      \[(\text{hhv. } f\mapsto F(f):
      \Hom_\Ccal(A,B)\to\Hom_\Dcal(F(B),F(A)))\]
  \end{enumerate}
  med følgende egenskaper
  \begin{enumerate}
    \item for ethvert objekt $A\in\Ob(\Ccal)$ har vi
      $F(\id_A) = \id_{F(A)}$
    \item for enhver
      $f\in\Hom_\Ccal(A,B),g\in\Hom_\Ccal(B,C)$ har vi
      \[F(g\circ f) = F(g)\circ F(f)\quad (\text{hhv.
      } F(g\circ f) = F(f)\circ F(f)).\]
  \end{enumerate}

\end{definisjon}

%\begin{definisjon}\label{Def:Funktor}
%    La $\Ccal$ og $\Dcal$ være kategorier en funktor $F:\Ccal\to\Dcal$ er funksjoner $\Ob(\Ccal)\to\Ob(\Dcal)$ og
%    $\hom(\Ccal)\to \hom(\Dcal)$ slik at for morfier $f: A\to B$ og $g: B\to C$ i $\Ccal$ så er $F(f): F(A)\to
%    F(B)$, $F(g\circ f) = F(g)\circ F(f)$ og for ethvert objekt $A\in\Ob(\Ccal)$ så er $F(\id_A)=\id_{F(A)}$.
%\end{definisjon}

%\begin{definisjon}\label{def:KontraFunktor}
%  En funktor på formen $F:\Ccal^\op\to\Dcal$ eller
%  $F:\Ccal\to\Dcal^\op$ er kalt en kontravariant funktor fra
%  $\Ccal$ til $\Dcal$.
%\end{definisjon}

Ofte skriver vi bare $F:\Ccal\to\Dcal$ for en kontravariant
funktor og sier at den er kontravariant. En kovariant
funktor kalles ofte bare en funktor.

\begin{bemerk}\label{Rem:FunktorIso}
    Gitt en funktor $F:\Ccal\to\Dcal$ og en isomorfi $f:A\to B$ i $\Ccal$ så er $F(f)$ en isomorfi i $\Dcal$. Siden $f$ er en isomorfi så eksisterer det en $g: B\to A$ med egenskapene i \bref{Definisjon}{Def:Iso}. Dette gir 
    \[\id_{F(A)} = F(\id_A) = F(f\circ g) = F(f)\circ F(g).\]
    På samme måte får vi $F(f\circ g) = \id_{F(B)}$. Dermed blir $F(g)$ en invers av $F(f)$ som betyr at den er en isomorfi.
\end{bemerk}

\begin{eksempel}\label{Ex:DiskFunkt}
    Vi kan definere en funktor $F:\Set\to\Top$ som setter den diskrete topologien på en mengde $X$, her blir alle funksjoner $f: X\to Y$ sendt til $F(f)=f:(X,\Pcal(X))\to (Y,\Pcal(Y))$, den er kontinuerlig siden for enhver delmengde $A\in \Pcal(Y)$ så er $f^{-1}(A)\in\Pcal(X)$.
\end{eksempel}
På samme måte har vi en funktor $F:\Set\to\Top$ som setter den trivielle topologien på en mengde og bevarer funksjonene.

\begin{eksempel}\label{Ex:Glemmefunktor}
    Vi har også en funktor $F:\Top\to\Set$ definert ved $F((X,\Tcal)) = X$ og $F(f:(X,\Tcal)\to (Y,\Tcal')) = f: X\to Y$. Denne funktoren kaller vi for glemmefunktoren siden den glemmer all struktur til rommet. En kan også gjøre dette for andre kategorier som $\Vect_K$ og $\Rbld$.
\end{eksempel}

\begin{eksempel}\label{ex:Homfunkt}
  Et eksempel på en kontravariant funktor er
  $\Hom(\cdot,W):\Vect_K\to\Vect_K$ som sender et vektorrom
  $V\in\Vect_K$ til vektorrommet $\Hom(V,W)$ av lineære
  avbildinger $V\to W$ og den sender lineære avbildinger $T:U\to
  V$ til $\Hom(T,W):\Hom(V,W)\to\Hom(U,W)$ via
  $\Hom(T,W)(g)=g\circ T$ for en $g:V\to W$. Her kan man se at
  funktoren snur på morfien som gjør den kontravariant.
\end{eksempel}

Man kan også definere en pil mellom to funktorer som gjort
i \citep{Agore2023}
\begin{definisjon}\label{def:NatTrans}
  La $F,G:\Ccal\to\Dcal$ være to funktorer. En naturlig
  transformasjon $\alpha: F\to G$ består av en familie av
  morfier $\alpha_C: F(C)\to G(C)$ i $\Dcal$ indisert av
  $C\Ob(\Ccal)$ slik at for enhver $f\in\Hom_\Ccal(C,C')$
  har vi $\alpha_{C'}\circ F(f) = G(f)\circ\alpha_C$.
\end{definisjon}

%\begin{definisjon}\label{Def:NatTrans}
%    La $\Ccal$ og $\Dcal$ være kategorier og la $F,G: \Ccal\to\Dcal$ funktorer. En naturlig transformasjoner er en pil $\eta: F\to G$ slik at for ethvert objekt $A\in \Ccal$ har vi en morfi $\eta_A: F(A)\to G(A)$ og for enhver morfi $f: A\to B$ i $\Ccal$ har vi at diagrammet
%    \[\begin{tikzcd}
%	{F(A)} && {G(A)} \\
%	\\
%	{F(B)} && {G(B)}
%	\arrow["{\eta_A}", from=1-1, to=1-3]
%	\arrow["{F(f)}"', from=1-1, to=3-1]
%	\arrow["{G(f)}"', from=1-3, to=3-3]
%	\arrow["{\eta_B}", from=3-1, to=3-3]
%\end{tikzcd}\]
%kommuterer.
%\end{definisjon}

Funktorer og naturlige transformasjoner gir oss en ny type kategori 
\begin{definisjon}\label{Def:FunkKat}
    For kategorier $\Ccal$ og $\Dcal$ er kategorien
    $\Ccal^\Dcal$, med funktorer $F:\Ccal\to \Dcal$ som
    objekter og naturlige transformasjoner som morfier,
    kalt en funktorkategori.
\end{definisjon}

Diagrammer er viktige for å kunne studere forskjellige egenskaper til objekter og morfier i en kategori. Her er
definisjonen på et diagram. I \citep[definisjon
1.4.9]{Agore2023} har vi definisjonen

\begin{definisjon}\label{def:Diagram}
   La $\Ccal$ være en kategori. Et diagram i $\Ccal$ er en
   graf med noder og kanter objekter og morfier i $\Ccal$
   hhv. Et diagram kalles kommutativt hvis for ethvert
   par av noder alle stier mellom dem er like som morfier.
\end{definisjon}

%\begin{definisjon}\label{def:Diagram}
%  La $\Jcal$ være en liten kategori. Et diagram av form $\Jcal$
%  i en kategori $\Ccal$ er en funktor $D:\Jcal\to\Ccal$.
%\end{definisjon}
%Dette er ikke en intuitiv definisjon, men diagrammer er ikke så
%farlige som en skal tro her er noen illustrative eksempler

Her er noen diagrammer i en gitt kategori fra
\citep[eksempler 1.4.10]{Agore2023}

\begin{eksempel}\label{eks:Diag1}
  Diagrammet
  \[\begin{tikzcd}
	A && B \\
	\\
	&& C
	\arrow["f", from=1-1, to=1-3]
	\arrow["h"', from=1-1, to=3-3]
	\arrow["g", from=1-3, to=3-3]
  \end{tikzcd}\]
  kommuterer når $h = g\circ f$
\end{eksempel}

\begin{eksempel}\label{eks:Diag2}
   Diagrammet
   \[\begin{tikzcd}
     A && B \\
     \\
     C && D
     \arrow["f", from=1-1, to=1-3]
     \arrow["k"', from=1-1, to=3-1]
     \arrow["g", from=1-3, to=3-3]
     \arrow["h"', from=3-1, to=3-3]
   \end{tikzcd}\]
   kommuterer når $g\circ f = h\circ k$.
\end{eksempel}

%\begin{eksempel}\label{ex:Diag1}
%  Her er en illustrasjon som viser hvordan formen på kategorien
%  $\Jcal$ gir et diagram i en kategori $\Ccal$
%
%  \[\begin{tikzcd}
%	& \Jcal &&&&&& \Ccal \\
%	\\
%	1 && 2 &&&& A && B \\
%	&&&& {\stackrel{D}{\implies}} \\
%	3 && 4 &&&& C && D
%	\arrow["{\id_1}", from=3-1, to=3-1, loop, in=100, out=170, distance=10mm]
%	\arrow["a", from=3-1, to=3-3]
%	\arrow["d", from=3-1, to=5-1]
%	\arrow["{\id_2}", from=3-3, to=3-3, loop, in=10, out=80, distance=10mm]
%	\arrow["b", from=3-3, to=5-3]
%	\arrow["{\id_A}", from=3-7, to=3-7, loop, in=100, out=170, distance=10mm]
%	\arrow["f", from=3-7, to=3-9]
%	\arrow["i", from=3-7, to=5-7]
%	\arrow["{\id_B}", from=3-9, to=3-9, loop, in=10, out=80, distance=10mm]
%	\arrow["g", from=3-9, to=5-9]
%	\arrow["{\id_3}", from=5-1, to=5-1, loop, in=190, out=260, distance=10mm]
%	\arrow["c", from=5-1, to=5-3]
%	\arrow["{\id_4}", from=5-3, to=5-3, loop, in=280, out=350, distance=10mm]
%	\arrow["{\id_C}", from=5-7, to=5-7, loop, in=190, out=260, distance=10mm]
%	\arrow["h", from=5-7, to=5-9]
%	\arrow["{\id_D}", from=5-9, to=5-9, loop, in=280, out=350, distance=10mm]
%\end{tikzcd}\]

%Funktoren $D$ sender $1$ til $A$, $2$ til $B$ osv. Siden $D$ er en funktor
%  sender den $\id_1$ til $\id_{D(1)}=\id_A$ osv. Funktoren sender
%  morfiene til de naturlige morfiene.
%\end{eksempel}
%
%\begin{eksempel}\label{ex:Diag2}
%  I dette eksempelet er $\Jcal$ trekantformet, dette gir
%  diagrammet $D:\Jcal\to\Ccal$
%
%  \[\begin{tikzcd}
%	& \Jcal &&&&& \Ccal \\
%	\\
%	1 && 2 &&& A && B \\
%	&&&& {\stackrel{D}{\implies}} \\
%	&& 3 &&&&& C
%	\arrow["{\id_1}", from=3-1, to=3-1, loop, in=100, out=170, distance=10mm]
%	\arrow["a", from=3-1, to=3-3]
%	\arrow["c", from=3-1, to=5-3]
%	\arrow["{\id_2}", from=3-3, to=3-3, loop, in=10, out=80, distance=10mm]
%	\arrow["b", from=3-3, to=5-3]
%	\arrow["{\id_A}", from=3-6, to=3-6, loop, in=100, out=170, distance=10mm]
%	\arrow["f", from=3-6, to=3-8]
%	\arrow["h", from=3-6, to=5-8]
%	\arrow["{\id_B}", from=3-8, to=3-8, loop, in=10, out=80, distance=10mm]
%	\arrow["g", from=3-8, to=5-8]
%	\arrow["{\id_3}", from=5-3, to=5-3, loop, in=280, out=350, distance=10mm]
%	\arrow["{\id_C}", from=5-8, to=5-8, loop, in=280, out=350, distance=10mm]
%\end{tikzcd}\]
%
%Funktoren $D$ sender $1$ til $A$, $2$ til $B$ og $3$ til $C$ og
%  morfiene blir sendt til de naturlig morfiene.
%\end{eksempel}
%Vi pleier aldri å faktisk tegne opp $\Jcal$ og funktoren $D$,
%definisjonen er bare nødvendig for å være presis. Objektenes
%identiteter er alltid tilstede i et diagram, derfor er det ikke nødvendig
%å tegne dem i diagrammet med mindre de er viktige i diagrammet.

%Vi bruker diagrammer hele tiden i matematikk og spesielt et type
%diagram kalt et kommutativt diagram
%
%\begin{definisjon}\label{def:KomDiag}
%Et diagram $D:\Jcal\to\Ccal$ er kalt kommutativt hvis alle morfier
%som starter med samme objekt og slutter på samme objekt er like.
%\end{definisjon}

%I \bref{eksempel}{ex:Diag1} er diagrammet kommutativt hvis $g\circ
%f = h\circ i$ og i \bref{eksempel}{ex:Diag2} er diagrammet
%kommutativt hvis $h=g\circ f$.


I en kategori med nullobjekt har morfier en kjerne og kokjerne.
Her er definisjonene på kjernen fra \cite[seksjon VIII.1]{Lane2010}

\begin{definisjon}\label{def:kjernen}
    I en kategori $\Ccal$ med nullobjekt $0$, la $f: A\to B$ være en morfi. Kjernen til $f$ er et objekt $K$ sammen med en morfi $k: K\to A$ slik at $f\circ k=0$ og gitt ethvert objekt $Q$ med morfi $h: Q\to A$ slik at $f\circ h=0$, eksisterer det en unik morfi $u: Q\to K$ slik at diagrammet
    \[\begin{tikzcd}
	K \\
	&& A && B \\
	Q
	\arrow["k", from=1-1, to=2-3]
	\arrow["0", curve={height=-12pt}, from=1-1, to=2-5]
	\arrow["f", from=2-3, to=2-5]
	\arrow["{\exists!\,u}", dashed, from=3-1, to=1-1]
	\arrow["h", from=3-1, to=2-3]
	\arrow["0"', curve={height=12pt}, from=3-1, to=2-5]
\end{tikzcd}\]
kommuterer. 
\end{definisjon}

\begin{proposisjon}
    I $\Vect_K$ er kjernen av en lineær avbilding $f: V\to
    W$ gitt ved $U = \bset{v\in V}{f(v)=0}$ og inkusjonen
    $k:U\to V$.
\end{proposisjon}
\begin{proof}[bevis]
La $Q$ være et vektorrom med lineær avbilding $h: Q\to V$
  med $f\circ h=0$. Vi må finne en $u:Q\to U$ slik at
  $k\circ u=h$. Siden $f\circ h=0$, har vi for enhver
  vektor $v\in Q$ er $f(h(v))=0$ altså er $h(v)\in U$.
  Dermed er $h=k\circ h'$ med $h':Q\to K$ gitt ved
  $h'(v)=h(v)$.
\end{proof}

Fra \citep[seksjon III.3]{Lane2010} har vi også kokjernen
av en morfi.

\begin{definisjon}\label{def:kokjernen}
  I en kategori $\Ccal$ med nullobjekt, er kokjernen av en
  morfi $f: A\to B$ en morfi $u: B\to E$ slik at
  \begin{enumerate}
    \item $u\circ f = 0: A\to E$
    \item hvis $h: B\to C$ har $h\circ f = 0$, da er
      $h=h'\circ u$ for en unik morfi $h':E\to C$.
  \end{enumerate}
\end{definisjon}

Fra \citep[definisjon 1.3]{Bauer2020} har vi følgende
viktig teorem for oppgaven

\begin{definisjon}\label{def:trivialitet}
   For $\Abld$ en kategori med nullobjekt og
   $\delta\geq0$, sier vi at et diagram $M:\Rbld\to\Abld$
   er $\delta$-triviell hvis for enhver $t\in\Rbb$, den
   interne morfien $M_{t,t+\delta}: M_t\to M_{t+\delta}$
   er en nullmorfi, altså den faktoriserer gjennom
   nullobjektet.
\end{definisjon}

Kategoriene i denne oppgaven er kalt Puppe-eksakte
kategorier. Definisjonen er gitt i \citep[definisjon
2.1]{Bauer2020}.

\begin{definisjon}\label{def:Puppe-eksakt}
   En Puppe-eksakt kategori er en kategori med følgende
   egenskaper:
  \begin{enumerate}
    \item den har et nullobjekt
    \item den har alle kjerner og kokjerner
    \item alle monomorfi er en kjerne og alle epimorfier
      er en kokjerne
    \item alle morfier har en epi-mono faktorisering.
  \end{enumerate}
\end{definisjon}

\subsubsection{Homologi}
Topologiske rom kan være vanskelige å studere i seg selv.
Heldigvis kan man overføre informasjon om rommet til en algebraisk
struktur som er lettere å håndtere. Vi kan gjøre dette på
forskjellige måter, men i denne oppgaven bruker vi
homologi. 

Følgende definisjoner er fra \citep[seksjon
2.1]{Hatcher2001}.

\begin{definisjon}\label{def:KjdKomp}
  Gitt et simplisialkompleks $X$, la $\Delta_n(X)$ være
  det frie vektorrommet over en kropp $K$ med alle
  $n$-simpleksene i $X$ som basis.
\end{definisjon}

Vektorer i $\Delta_n(X)$ er formelle summer $\sum
n_\alpha\sigma_\alpha$ hvor $n_\alpha\in K$ og
$\sigma_\alpha\in X$ et $n$-simpleks. Vi kaller vektorer
i $\Delta_n(X)$ for $n$-kjeder. 

\begin{definisjon}\label{def:randhom}
  For ethvert simplisialkompleks og enhver $n\in\Zbb^+$
  har vi avbildingen 
  \[\partial_n: \Delta_n(X)\to\Delta_{n-1}(X)\]
  som sender et $n$-simpleks
  \[[v_0,\dots,v_n]\mapsto
  \sum_{i=0}^n(-1)^i[v_0,\dots,\hat{v}_i,\dots,v_n]\]
  Hvor $[v_0,\dots,\hat{v}_i,\dots,v_n]$ er
  $(n-1)$-simplekset hvor vi sletter $v_i$. Avbildingene
  $\partial_n$ er kalt randhomomorfier.
\end{definisjon}

\begin{eksempel}\label{eks:randhom}
  For simplisialkomplekset i \bref{eksempel}{eks:Sirkel}
  har vi
  \[\Delta_0(X) = \bset{a[p_0]+b[p_1]+c[p_2]}{a,b,c\in K}\]
  \[\Delta_1(X)
  = \bset{a[p_0,p_1]+b[p_0,p_2]+c[p_1,p_2]}{a,b,c\in K}\]
  \[\Delta_n(X) = \{0\},\quad n>1\text{ eller } n<0\]
  og randhomomorfiene blir
  \[\partial_1([p_i,p_j]) = p_j-p_i\]
  for ethvert par $i,j=0,1$ og
  \[\partial_n = 0.\]
\end{eksempel}

Fra definisjonen av randhomomorfiene har vi at
$\partial_n\circ\partial_{n+1}=0$. Dette betyr at
$\im\partial_{n+1}\subset\ker\partial_n$.
Definisjonen på homologien av et rom er gitt ved
simplisialkomplekset til rommet. I \citep[seksjon
2.1]{Hatcher2001} definerer forfatter den $n$-te
  homologigruppen ved

\begin{definisjon}\label{def:Homologi}
  Den $n$-te homologien av et topologisk rom med
  simplisialkompleks $X$ er definert
  som $H_n(X) = \ker\partial_n/\im\partial_{n+1}$.
\end{definisjon}

I vårt tilfellet er $H_n(X)$ et vektorrom over en kropp
$K$.

