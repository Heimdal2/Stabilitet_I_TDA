\subsection{Multimengder og parvis kobling}

Mengder er ofte begrensende i og med at de ikke tillater
repeterende elementer.

\begin{definisjon}\label{def:MultMengde}
  En samling med (potensielt) repeterende elementer er
  kalt en multimengde. 
\end{definisjon}

For å differensiere mellom vanlige mengder og multimengder
bruker vi en annen notasjon for multimengder

\begin{definisjon}\label{def:Mult_Notasjon}
    En multimengde $A$ av elementer som tilfredstiller et
    predikat $\Phi$ er skrevet som følger
    \[\bbag{x}{\Phi(x)}\]
\end{definisjon}

\begin{definisjon}\label{def:Mult_Under}
  For enhver multimengde $\Scal$ eksisterer det en unik
  mengde $S$ som ignorerer repetisjoner. Det eksisterer
  også en funksjon $\mu: S\to\mathbb{N}$ som teller
  multiplisiteten av et element in $\Scal$
\end{definisjon}

\begin{definisjon}\label{def:Rep}
   Representasjonen av en multimengde $\Scal$ er en mengde
   definert ved
   \[\Rep(\Scal) = \bset{(x,k)}{x\in S, k\leq \mu(x)}\]
\end{definisjon}

mellom to mengder kan vi definere noe kalt en parvis
kobling

\begin{definisjon}\label{def:Parvis-Kobling}
  En parvis kobling mellom mengder $S$ og $T$ (skrevet
  $\sigma:S\to T$) er en bijeksjon $\sigma: S'\to T'$
  mellom delmengder $S'\subset S$ og $T'\subset T$.
  Formelt er $\sigma\subset S\times T$ en relasjon slik at
  $(s,t)\in \sigma$ hvis og bare hvis $s\in S'$ og
  $\sigma(s)=t$. Komposisjonen av to parvis koblinger
  $\sigma: S\to T$ og $\tau: T\to U$ er definert som
  relasjonen
  \[\tau\circ\sigma=\bset{(s,u)}{(s,t)\in \sigma, (t,u)\in
  \tau\;\;\text{for en $t\in T$}}.\]
\end{definisjon}

Klassen av mengder sammen med parvis koblinger gir en ny
kategori vi kaller \Mch.
