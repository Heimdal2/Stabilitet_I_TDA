\subsection{Topologiske rom}\label{Sek:TopRom}
Topologiske rom generaliserer metriske rom videre ved
å kvitte seg med en metrikk som måler avstanden mellom
punkter i rommet. I topologiske rom er alt beskrevet ved
åpne mengder. Vi bruker topologiske rom til å undersøke
formen på data. Definisjonen som følger er tatt fra
\citep[seksjon 12]{Munkres2013}. 

\begin{definisjon}\label{Def:TopRom}
    Et par $(X,\Tcal)$ hvor $X$ er en mengde og
    $\Tcal\subset\Pcal(X)$ slik at 
    \begin{itemize}
        \item $X$ og $\es$ er i $\Tcal$.
        \item Unionen av elementene i enhver delmengde
          $\Tcal$ er også i $\Tcal$.
        \item Snittet av elementene i enhver endelig
          delmengde av $\Tcal$ er i $\Tcal$.
    \end{itemize}
    Vi kaller mengden $\Tcal$ for topologien på $X$ og
    mengdene i $\Tcal$ for åpne mengder.
\end{definisjon}

Mellom topologiske rom har vi også en spesiell type
funksjon kalt en kontinuerlig funksjon. Definisjonen på en
kontinuerlig funksjon er gitt i \citep[seksjon 18]{Munkres2013}

\begin{definisjon}\label{Def:KontFunk}
    La $(X,\Tcal_X)$ og $(Y,\Tcal_Y)$ være topologiske
    rom. En funksjon $f: X\to Y$ er kalt kontinuerlig hvis
    for en hver \(V\in\Tcal_Y\) så er
    \(f^{-1}(V)\in\Tcal_X\).
\end{definisjon}

\begin{eksempel}\label{Ex:KontFunk}
    La $f: \Rbld\to \Rbld$ være funksjonen $f(x)=2x+1$, da
    er $f$ kontinuerlig. La $V = (a,b)$ et opent
    intervall, da blir $f^{-1}(V)
    = \{x\in\Rbld\;|\;2x+1\in
    V\}=\bp{\frac{a}{2}-1,\frac{b}{2}-1}$ som også er et
    åpent intervall.
\end{eksempel}

Når topologien $\Tcal$ på en mengde $X$ er kjent eller
ikke viktig lar vi være å skrive det topologiske rommet
som et par $(X,\Tcal)$ og skriver bare $X$. Alle
funksjoner mellom topologiske rom vil være kontinuerlige.
Til slutt så vil et "rom" bety et topologisk rom.

\begin{eksempel}\label{Ex:EukTRom}
    Euklidisk rom $(\Rbld^n,\Tcal)$ er et topologisk rom
    med åpne mengder unioner av vilkårlig mange mengder av
    typen
    \[\Bcal(x,\delta) = \{y\in\Rbld^n\;|\;
    \|x-y\|<\delta\}\]
    kalt åpne baller. Euklidisk rom er som regel alltid
    bare skrevet $\Rbld^n$ siden det er den topologien på
    $\Rbld^n$ som er antatt.
\end{eksempel}

For et topologisk rom $(X,\Tcal)$ og en delmengde
$A\subset X$ er det en naturlig topologi vi kan sette på
$A$.

\begin{definisjon}\label{Def:UnderTop}
    La $(X,\Tcal)$ være et topologisk rom og la $A\subset
    X$ da er det en naturlig topologi $\Tcal_A$ vi kan
    sette på $A$ definert ved
    \[U\in\Tcal_A \text{ hvis og bare hvis det eksisterer
    en åpen mengde } V\in\Tcal\quad\text{slik at}\quad
    V\cap A = U\]
    Vi kaller $(A,\Tcal_A)$ et underrom av $(X,\Tcal)$ og
    vi kaller $\Tcal_A$ underromstopologien på $A$.
\end{definisjon}


\subsubsection{Homotopi}\label{sec:Homotopi}
Fra \citep[seksjon 51]{Munkres2013} kan man tenke at
homotopi representerer en kontinuerlig "deformering" fra en
funksjon til en annen.

\begin{definisjon}\label{Def:Homotopi}
    La $X$ og $Y$ være topologiske rom og la $f,g: X\to Y$
    være funksjoner. En homotopi mellom $f$ og $g$ er en
    funksjon
    \[F: X\times[0,1]\to Y.\]
    Slik at $F(x,0)=f(x)$ og $F(x,1)=g(x)$.
    Hvis det eksisterer en homotopi mellom en funksjon $f$
    og $g$ sier vi at de er homotope og vi skriver
    $f\simeq g$.
\end{definisjon} 

\begin{definisjon}\label{Def:HomotopiEkv}
    To topologiske rom $X$ og $Y$ er homotopiekvivalente
    hvis det eksisterer funksjoner $f: X\to Y$ og $g: Y\to
    X$ slik at
    \[g\circ f\simeq\id_X\quad f\circ g \simeq \id_Y\]
\end{definisjon}

\subsubsection{Simplisialkomplekser}\label{sek:SimpKomp}
Gitt at denne og neste seksjon om homologi er viktige
i \bref{seksjon}{sek:VarHom}, men kommer ikke til å bli
brukt ellers vil disse seksjonene inneholde uformelle
forklaringer og noen eksempler. Definisjonene er tatt fra
\citep[seksjon 1]{MunkresJamesR.2018EOAT} og
\citep[seksjon 2]{MunkresJamesR.2018EOAT}.

\begin{definisjon}\label{def:GeoUav}
  En mengde $\set{a_0,\dots,a_n}$ av punkt i $\Rbb^N$ er
  geometrisk uavhengige hvis likningene
  \[\sum_{i=0}^nt_i
  = 0\quad\text{og}\quad\sum_{i=0}^nt_ia_i=0\]
  impliserer at $t_0=t_1=\dots=t_n=0$.
\end{definisjon}

\begin{definisjon}\label{def:n-simp} 
  La $\set{a_0,\dots,a_n}$ være en geometrisk uavhengig mengde
  i $\Rbb^N$. Vi lar $n$-simplekset $\sigma$ utspent av
  $a_0,\dots,a_n$ være mengden av punkt
  \[\sum_{i=0}^nt_ia_i\quad\text{hvor}\quad\sum_{i=0}^nt_i = 1\]
  og $t_i\geq0$ for alle $i$. Vi skriver $[a_0,\dots,a_n]$ for
  simplekset utspent av $\set{a_0,\dots,a_n}$.
\end{definisjon}

\begin{definisjon}\label{def:SimpKomp}
  Et simplisialkompleks $K$ i $\Rbb^N$ er en samling av simplekser
  i $\Rbb^N$ slik at:
  \begin{enumerate}
    \item En side av et simpleks av $K$ er i $K$.\\
    \item Snittet av ethvert par av simplekser av $K$ er en side
      av begge simpleksene.
  \end{enumerate}
\end{definisjon}

I \citep[seksjon 5]{MunkresJamesR.2018EOAT} gir forfatter følgende
definisjon om simplisialkomplekser

\begin{definisjon}\label{def:Orientering}
  La $\sigma$ være et simpleks. Definer to rekkefølger av dens
  nodemengde som ekvivalente hvis de er forskjellig fra en annen
  opp til en jevn permutasjon. Hvis $\dim\sigma>0$, faller
  rekkefølgene av dens noder inn i en av to ekvivalensklasser.
  Enhver av disse klassene er kalt en orientasjon av $\sigma$. Et
  orientert simpleks er et simpleks med en orientasjon.
\end{definisjon}

\begin{eksempel}\label{eks:Sirkel}
  La $p_0,p_1,p_2$ være geometrisk uavhengige punkter
  i $\Rbb^2$ da er $X
  = \set{[p_0],[p_1],[p_2],[p_0,p_1],[p_0,p_2],[p_1,p_2]}$ et
  simplisialkompleks for $S^1$.
\end{eksempel}

% En måte å lage topologiske rom er å starte med enkle byggeklosser og lime dem sammen. Dette kan vi gjøre ved å bruke $n$-dimensjonale trekanter kalt $n$-simplekser.

%Definisjonene er fra \cite{Hatcher2002}

%\begin{definisjon}\label{Def:StndrdSimpKomp}
%    Standardsimplekset $\Delta^n$ er definert ved
%    \[\Delta^n = \{ \vct{x}\in\Rbld^{n + 1} \mid \sum_{i = 0}^{n + 1}x_i=1, x_i\geq 0\;\forall i=0,\dots,n\}\]
%\end{definisjon}

%En side på et $n$-simpleks er en $(n-1)$-simpleks.
%\begin{definisjon}\label{Def:Deltastrkt}
%    Et $\Delta$-kompleks struktur på et rom $X$ er en samling av kontinuerlige funksjoner $\sigma_\alpha: \Delta^n\to X$ med $n$ avhengig av $\alpha$ slik at:
%    \begin{enumerate}
%        \item Restriksjonen $\sigma_\alpha\mid\mathring{\Delta}^n$ er injektiv og hvert punkt i $X$ er i bildet av nøyaktig en slik Restriksjon.
%        \item Hver restriksjon av $\sigma_\alpha$ til en side av $\Delta^n$ er en av funksjonene $\sigma_\beta: \Delta^{n-1}\to X$
%        \item En mengde $A\subset X$ er åpen hvis og bare hvis $\sigma_\alpha^{-1}(A)$ er åpen for hver $\sigma_\alpha$.
%    \end{enumerate}
%\end{definisjon}
%Disse kriteriene gir oss en måte å lage forskjellige topologiske rom ved enkle byggeklosser.
%
%\begin{eksempel}[ex:SirkelDkomp]
%    Vi kan lage et $\Delta$ for sirkelen $S^1$ ved følgende:
%    \begin{itemize}
%        \item La $\sigma_0:\Delta^0=\{\ast\}\to S^1$ være en avbilding som sender $\ast$ til et punkt i $S^1$.
%        \item La $\sigma_1: \Delta^1\to X$ være avbildingen som sender $\partial \Delta^1$ til punktet $\sigma_0(\ast)$ og alle andre punkter $x\in\inter{\Delta}^1$ sendes injektiv til $S^1$.
%    \end{itemize}
%    Her er $\{\sigma_0,\sigma_1\}$ et $\Delta$-kompleks på $S^1$. Intuitivt kan en tenke på $\sigma_1$ som at man limer fast endepunktene til linjen som gir oss sirkelen.
%\end{eksempel}
