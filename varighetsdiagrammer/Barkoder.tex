\section{Strekkoder}
En barkode $\Bcal$ er en representasjon av en multimengde
av intervaller. Elementer i en barkode er dermed par
$(I,k)$ der $I$ er et intervall og $k\in\Nbb$. Ofte når
indeksen $k$ er nødvendig skriver vi bare $I$ for et
intervall i barkoden.

I \citep[teorem 2.4]{Bauer2018} har vi følgende resultat.
\begin{teorem}\label{trm:DekompBark}
  For enhver p.e.d. varighetsmodul $M$, eksisterer det en
  unik naturlig strekkode $\Bcal(M)$ slik at
  \[M\cong \bigoplus_{I\in\Bcal(M)} K^I.\]
\end{teorem}
Forfatter kaller strekkoden $\Bcal(M)$
dekomponeringsstrekkoden.

%I \cite{Bauer2015} skriver forfatter at gitt en
%varighetsmodul $M$ som kan skrives på formen
%\[M\cong\bigoplus_{I\in\Bcal_M}K^I,\]
%hvor $\Bcal_M$ er en barkode.
%Da er $\Bcal_M$ unikt bestemt. Vi kaller slike varighetsmoduler intervalldekomponerbare.
%
%I \cite{Bauer2015a} har vi følgende teorem
%\begin{teorem}\label{Thrm:thrm2.1}
%Enhver p.e.d. varighetsmodul er intervalldekomponerbar.
%\end{teorem}

Strekkoder kan også gjøres om til en kategori med morfier
vi kaller overlappkoblinger. For å definere en overlappkobling
må vi først definere hva det betyr at et intervall
overlapper et annet intervall over og hva en parvis
kobling av mengder er. Fra \citep[seksjon 2.3]{Bauer2018}
har vi følgende definisjon.

\begin{definisjon}
    Et intervall $I$ overlapper et annet intervall $J$
    over (hhv. overlapper $J$  $I$ under) hvis følgende
    holder
    \begin{itemize}
        \item $I\cap J\neq\es$.
        \item For enhver $s\in J$ eksisterer det en $t\in I$ slik at $t\leq s$. Vi sier at $I$ begrenser $J$ over.
        \item For enhver $s\in I$ eksisterer det en $t\in J$ slik at $t\leq s$. Vi sier at $J$ begrenser $I$ under.
    \end{itemize}
\end{definisjon}

\begin{definisjon}\label{def:OverlappMch}
  En parvis kobling $\sigma: \Ccal\to \Dcal$ av strekkoder
  er en overlappkobling hvis gitt $\sigma(I)=J$ så
  overlapper $I$ $J$ over.
\end{definisjon}

Komposisjonen av overlappkoblinger som parvis koblinger
resulterer ikke alltid i en overlappkobling. Vi endrer
komposisjonen på følgende måte

\begin{definisjon}\label{def:OK-Komp}
    La $\sigma: \Bcal\to \Ccal$ og $\tau:\Ccal\to\Dcal$
    være overlappkoblinger. Komposisjonen blir da
    definert som
    \[\tau\bullet\sigma
      = \bset{(I,J)\in\tau\circ\sigma}{\text{$I$
    overlapper $J$}}\]
    Her er $\tau\circ\sigma$ komposisjonen som parvis
    koblinger.
\end{definisjon}

\begin{proposisjon}\label{prop:ID-OK}
For enhver strekkode $\Dcal$ eksisterer det alltid en
overlappkobling $\sigma:\Dcal\to\Dcal$ gitt ved $\sigma
= \bset{(I,I)}{I\in\Dcal}$.
\end{proposisjon}
\bevis{
Første punktet er å vise at $I\cap\sigma(I)\neq\es$. Dette
er sant siden $\sigma(I)=I$ altså er $I\cap\sigma(I)=I\cap
I$ som er lik $I\neq\es$. Andre punktet er å vise at $I$
overlapper seg selv over. Velg en $s\in I$ da har vi
$s\leq s$ dermed begrenser $I$ seg selv over og under.
Dette betyr at $\sigma$ er en overlappkobling.
}

\begin{teorem}
    Klassen av strekkoder sammen med overlappkoblinger
    skaper en kategori.
\end{teorem}
\bevis{
Objektene er strekkodene og morfiene er overlappkoblingene.
Vi lar identiteten til en strekkode være overlappkoblingen
$\id_\Dcal:\Dcal\to\Dcal$ gitt
i \bref{proposisjon}{prop:ID-OK} Komposisjonen er den
definert i \bref{definisjon}{def:OK-Komp}. Vi må vise at
$\bullet$ komposisjonen er assosiativ og respekterer
identiteter.

Vi starter med identitetene. La $\sigma:
\Dcal\to\Ccal$ være en overlappkobling da har vi
\[\sigma\bullet\id_\Ccal = \bset{(I,J)\in\sigma\circ\id_\Ccal}{\text{$I$ overlapper $J$ over}}.\]
La $(I,J)\in \sigma\bullet\id_\Ccal$, Da eksisterer det en
$K\in\Ccal$ slik at $(I,K)\in\id_\Ccal$ og
$(K,J)\in\sigma$ og $I$ overlapper $J$ over. Siden
$\id_\Ccal(I)=I$ så må $K=I$ altså er
$\sigma(K)=\sigma(I)=J$, derfor har vi at
$(I,J)\in\sigma$. Dette betyr at
$\sigma\bullet\id_\Ccal\subset\sigma$ og gitt et par
$(I,J)\in\sigma$ har vi paret $(I,I)\in\id_\Ccal$ slik at
$J = \sigma(I) = \sigma\bullet\id_\Ccal(I)$ dermed har vi
også at $\sigma\subset\sigma\bullet\id_\Ccal$ det vil si
at $\sigma=\sigma\bullet\id_\Ccal$. På samme måte kan vi
vise at $\sigma = \id_\Dcal\bullet\sigma$.

La $\sigma:\Bcal\to\Ccal$, $\tau:\Ccal\to\Dcal$ og
$\psi:\Dcal\to\Ecal$. La $(I,K)$ være et par
i komposisjonen $(\psi\bullet\tau)\bullet\sigma$. Dette
betyr at det eksisterer et intervall $J\in\Ccal$ slik at
$(I,J)\in\sigma$ og $(J,K)\in\psi\bullet\tau$ og at $I$
overlapper $K$ over. Siden $(J,K)\in\psi\bullet\tau$ må
det også finnes et intervall $L\in\Dcal$ slik at
$(J,L)\in\tau$ og $(L,K)\in\psi$ og $J$ må overlappe $K$
over. Fordi $(I,J)\in\sigma$ og $(J,L)\in\tau$
}

\Barc er en Puppe-eksakt kategori. Dette betyr at kategorien har et nullobjekt gitt under.

\begin{proposisjon}\label{prop:Ønull}
  Den tomme mengden $\empty$ er nullobjektet i $\Barc$. 
\end{proposisjon}
\bevis{
  La $\Dcal$ være en strekkode. En overlappkobling $\sigma:
  \empty\to\Dcal$ er en delmengde av
  $\empty\times\Barc=\empty$, siden $\empty$ er sin eneste
  delmengde så er $\sigma=\empty$. For samme grunn er det
  bare en overlappkobling $\Dcal\to\empty$. Dermed er
  $\empty$ nullobjekt.
}

I \cite{Bauer2018} \textbf{seksjon 2.5} beskriver
forfatter kjernen, kokjernen og bildet til en
overlappkobling som strekkoder, men som beskrevet i seksjon
\bref{seksjon}{Ssek:Kategoriteori}
i \bref{definisjon}{def:kjernen} og
\bref{definisjon}{def:kokjernen} er ikke bare (ko)kjernen
av en morfi i en kategori et objekt, det er et par gitt av
et objekt og en morfi. Her er konstruksjonen av kjernen,
kokjernen og bildet, kjernen av kokjernen, som strekkoder
sammen med en overlappkobling.

\begin{teorem}\label{trm:ker}
  Kjernen til en overlappkobling $\sigma:\Ccal\to\Dcal$
  er en strekkode $\ker\sigma$ med overlappkobling
  $\kappa_\sigma:\ker\sigma\to\Ccal$
  \[\ker\sigma = \bbag{\ker(\sigma,
    I)\neq\empty}{I\in\Ccal},\quad\kappa_\sigma
  = \bset{(\ker(\sigma,I),I)}{\ker(\sigma,I)\neq\empty}\]
\end{teorem}
\bevis{
  Gitt en strekkode $\Bcal$ og en overlappkobling
  $\eta:\Bcal\to\Ccal$ slik at $\sigma\bullet\eta=\empty$,
  viser vi at det eksisterer en unik overlappkobling
  $\eta':\Bcal\to\ker\sigma$ slik at
  $\eta=\kappa_\sigma\bullet\eta'$. Vi antar at en slik
  $\eta'$ eksisterer og finner et uttrykk for
  overlappkobling. La $(I,J)\in\eta$ altså er
  $(I,J)\in\kappa_\sigma\bullet\eta'$. Fra
  \bref{definisjon}{OK-Komp} betyr dette at det
  eksisterer et intervall $K\in\Ccal$ slik at
  $(I,K)\in\eta'$ og $(K,J)\in\kappa_\sigma$. Siden
  $(K,J)\in\kappa_\sigma$ må $K = \ker(\sigma,J)$. Dermed
  har vi
  \[\eta'(I) = \ker(\sigma,\eta(I)).\]
  Her lar vi $\eta'$ koble $I\in\Bcal$ hvis $\eta$
  kobler $I$ og $I$ overlapper $\ker(\sigma,\eta(I))$
  over. Dermed for ethvert par
  $(\Bcal,\eta:\Bcal\to\Ccal)$ av en strekkode og
  overlappkobling slik at $\sigma\bullet\eta=\empty$,
  eksisterer det en unik overlappkobling
  $\eta':\Bcal\to\Ccal$ definert som over slik at
  $\kappa_\sigma\bullet\eta' = \eta$.
}

\begin{teorem}\label{trm:coker}
  Kokjernen til en overlappkobling er gitt som en
  strekkode $\coker\sigma$ med en overlappkobling
  $\mu_\sigma:\Dcal\to\coker\sigma$ gitt ved
  \[\coker\sigma
  = \bbag{\coker(\sigma,J)\neq\empty}{J\in\Dcal},\quad
  \mu_\sigma = \bbag{(J,\coker(\sigma,J))}{\text{$J$
  overlapper $\coker(\sigma,J)$ over}}\]
\end{teorem}
\bevis{
  Vi viser at gitt en strekkode $\Bcal$ og en
  overlappkobling $\eta:\Dcal\to\Bcal$ slik at
  $\eta\bullet\sigma=\empty$ eksisterer det en unik
  overlappkobling $\eta':\coker\sigma\to\Bcal$ slik at
  $\eta'\bullet\mu_\sigma=\eta$. Vi gjør dette på samme
  måte som i beviset for \bref{teorem}{trm:ker}. La
  $(I,J)\in\eta$ dermed er
  $(I,J)\in\eta'\bullet\mu_\sigma$. Siden
  $(I,J)\in\eta'\bullet\mu_\sigma$ eksisterer det et
  intervall $K\in\coker\sigma$ slik at
  $(I,K)\in\mu_\sigma$ og $(K,J)\in\eta'$. Siden
  $(I,K)\in\mu_\sigma$ er $K = \coker(\sigma,I)$. Dermed er
$(\coker(\sigma,I),J)\in\eta'$. Vi lar dermed
$\eta'(\coker(\sigma,I)) = J$ hvor $(I,J)\in\eta$ og
$\coker(\sigma,I)$ overlapper $J$ over.
}

\begin{teorem}\label{trm:im}
  Bildet av en overlappkobling $\sigma:\Ccal\to\Dcal$ er
  en strekkode $\im\sigma$ og en overlappkobling
  $\omega_\sigma:\im\sigma\to\Dcal$ gitt ved
  \[\im\sigma = \bbag{I\cap J}{(I,J)\in\sigma},\quad
  \omega_\sigma = \bbag{(I\cap J, J)}{(I,J)\in\sigma}\]
\end{teorem}
\bevis{
  Siden $\Barc$ er en Puppe-eksakt kategori er bildet av
  $\sigma$ kjernen av kokjernen av $\sigma$. Strekkoden blir
  multimengden av elementer på formen
  $J-(\coker(\sigma,J))\neq\empty$. Hvis $\sigma$ ikke
  parvis kobler $J$ er $J-(\coker(\sigma,J))=J-J=\empty$,
  hvis $(I,J)\in\sigma$, er $J-(\coker(\sigma,J))=J-(J-I)$
  som er det samme som $I\cap J$. Siden $(I,J)\in\sigma$
  overlapper $I$ $J$ over, altså er $I\cap J\neq\empty$
  som videre betyr at
  \[\ker\mu_\sigma = \im\sigma = \bbag{I\cap
  J}{(I,J)\in\sigma}.\]
  La $(L,J)\in\kappa_{\mu_\sigma}$ da er
  $L=\ker(\mu_\sigma,J)$. Dermed hvis $\mu_\sigma$ ikke
  parvis kobler $L$ er $L=J$. Hvis $(J,K)\in\mu_\sigma$
  er $K = \coker(\sigma,J)$ som er $J$ hvis $\sigma$ ikke
  parvis kobler $J$, da er $K=\empty$ og bryter
  overlapping, eller så er $(I,J)\in\sigma$ og $K = J-I$.
  I tilfellet der $(I,J)\in\sigma$ er $L=J-K=J-(J-I)$ som
  tilslutt er $I\cap J$. Dermed er $\omega_\sigma(I\cap
  J)=J$, eller sagt annerledes er $\omega_\sigma$
  multimengden av par $(I\cap,J,J)$ hvor $(I,J)\in\sigma$.
}

Som sagt i \bref{seksjon}{sek:VarMod} er det også en
avstand mellom strekkoder.

\subsection{Flaskehalsavstanden}
En kan definere en avstand av strekkoder på to
forskjellige måter gitt i \cite{Bauer2018}. En måte er
å definere en type inflettingsavstand i \Barc og en annen
kalt Flaskehalsavstanden. Vi starter med å definere den
første type avstand.

\begin{bemerk}\label{bem:strek_triv}
  Under \citep[definisjon 1.3]{Bauer2018} skriver
  forfatter at en strekkoder er $\delta$-triviell hvis og
  bare hvis alle intervallene i strekkoden er inneholdt
  i et halv-åpent intervall av lengde $\delta$.
\end{bemerk}

\begin{bemerk}\label{bem:int_triv}
  Et intervall er $\delta$-triviell hvis strekkoden
  $\{I\}$ er $\delta$-triviell
\end{bemerk}

I \citep[seksjon 2.5]{Bauer2018} definerer forfatter en
forskyvingsavbilding av intervaller som induserer en
funktor $\Barc\to\Barc$.
\begin{definisjon}\label{def:IntSkyv}
    For et intervall $I\subset\Rbb$ og $\delta\geq 0$ definerer vi
    \[I(\delta) = \bset{t}{t+\delta\in I}\]
    Dette intervallet er $I$ forskjøvet en $\delta$ til venstre.
\end{definisjon}

Dette gir en funktor $(\cdot)(\delta):\Barc\to\Barc$ definert ved

\begin{definisjon}
    For en strekkode $\Bcal$, lar vi
    \[\Bcal(\delta) = \bset{I(\delta)}{I\in\Bcal}.\]
    For en overlappkobling $\sigma:\Ccal\to\Dcal$ definerer vi
    \[\sigma(\delta) = \bset{(I(\delta),J(\delta))}{(I,J)\in\sigma}.\]
\end{definisjon}

Fra \cite{Bauer2018} har vi for enhver strekkode $\Ccal$
og $\delta\geq0$ overlappkoblingen $S^{\Ccal,\delta}
= \bbag{(I,I(\delta))}{\text{$I$ ikke er
$\delta$-triviell}}$.

\begin{definisjon}\label{def:InfBar}
To strekkoder $\Ccal$ og $\Dcal$ sies å være
  $\delta$-inflettet hvis det eksisterer overlappkoblinger
  $f:\Ccal\to \Dcal(\delta)$ og $g:\Dcal\to\Ccal(\delta)$
  slik at $g(\delta)\bullet f = S^{\Ccal,2\delta}$ og
  $f(\delta)\bullet g=S^{\Dcal,2\delta}$.
\end{definisjon}

Dette gir på samme måte som i \bref{seksjon}{sek:VarMod} en inflettingsavstand gitt ved
\[d_I(\Ccal,\Dcal) = \inf\bset{\delta\in\ropen{0}{\infty}}{\text{$\Ccal$ og $\Dcal$ $\delta$-inflettet}}.\]

Den andre måten en kan definere en avstand på er gitt i \cite[seksjon 4.2]{Bauer2018}.

La $U_\delta(I) = \bset{t\in\Rbb}{\text{det eksisterer en $s\in I$ slik at }|s-t|\leq\delta}$

\begin{definisjon}\label{def:delta-kobling}
En $\delta$-kobling $\sigma:\Ccal\to \Dcal$ er en parvis kobling, den trenger ikke å være en overlappkobling, med følgende egenskaper:
\begin{enumerate}
    \item $\sigma$ kobler hvert intervall i $\Ccal\cup\Dcal$ som ikke er $2\delta$-trivielle\\
    \item Hvis $(I,J)\in\sigma$, er $J\subset U_\delta(I)$ og $I\subset U_\delta(J)$.
\end{enumerate}
\end{definisjon}

\begin{definisjon}\label{def:FlaskAvst}
Gitt to strekkoder $\Ccal$ og $\Dcal$ er flaskehalsavstanden
\[d_B(\Ccal,\Dcal) = \inf\bset{\delta\geq 0}{\text{Det eksisterer en $\delta$-kobling mellom $\Ccal$ og $\Dcal$}}.\]
\end{definisjon}

Nå som vi har to typer avstander av strekkoder er det
naturlige spørsmålet "hvilken skal man bruke?" Heldigvis
er det ikke nødvendig å velge fordi inflettingsavstanden
og flaskehalsavstanden er like.

\begin{teorem}\label{trm:FA_lik_IA}
Gitt strekkoder $\Ccal$ og $\Dcal$ holder følgende
\[d_I(\Ccal,\Dcal) = d_B(\Ccal,\Dcal)\]
\end{teorem}
\bevis{

}
