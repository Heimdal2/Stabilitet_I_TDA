\section{Det induserte koblingsteoremet}
For å relatere flaskehalsavstanden av strekkoder og
inflettingsavstanden av varighetsmoduler må vi ha en måte
å relatere en $\delta$-inflettingsmorfi av
varighetsmoduler med en $\delta$-kobling av deres
strekkoder. Det induserte koblingsteoremet lar oss
relatere varighetsmodulmorfier til en korresponderende
overlappkobling. \citep[Teorem 3.1]{Bauer2020} gir
følgende resultat

\begin{teorem}\label{trm:IKT}
  La $f: M\to N$ være en varighetsmodulmorfi på p.e.d.
  varighetsmoduler.
  \begin{enumerate}
    \item Den induserte parvis koblingen
      $\chi(f):\Bcal(M)\to\Bcal(N)$ er en overlappkobling.
    \item Hvis $\ker f$ er $\delta$-triviell, da har vi
      \begin{enumerate}
        \item for ethvert par $(I,J)\in\chi(f)$, begrenser $J$
          $I(\delta)$ over og
        \item ethvert intervall i $\Bcal(M)$ som ikke er
          koblet av $\chi(f)$ er inneholdt i et halvåpent
          intervall av lengde $\delta$.
      \end{enumerate}

    \item Hvis $\coker f$ er $\delta$-triviell, da har vi
      \begin{enumerate}
        \item for ethvert par $(I,J)\in\chi(f)$, begrenser
          $I(\delta)$ $J$ under og
        \item for ethvert intervall $\Bcal(N)$ som ikke er
          koblet av $\chi(f)$ er inneholdt i et halvåpent
          intervall av lengde $\delta$.
      \end{enumerate}
  \end{enumerate}
\end{teorem}

For å bevise dette teoremet må vi først definere hva
$\chi(f)$ er for en varighetsmodul $f: M\to N$.

\subsection{Koblinger indusert av mono- og epimorfier av
varighetsmoduler}
I \citep[seksjon 3.2]{Bauer2020} starter forfatter med
å definere $\chi(f)$ for monomorfier og epimorfier av
varighetsmoduler. Dette gjøres ved å bruke indeksene til
intervallene i strekkodene. Forfatter definerer en
ekvivalensrelasjon $\sim_a$ på mengden av intervaller
$\Ical$ ved å la $I\sim_a J$ hvis $I$ og $J$ samfaller
over, altså $I$ begrenser $J$ over og $J$ begrenser $I$
over.

\begin{bemerk}
  La $I = \langle a,b)$ og $J = \langle a,b]$, da
  begrenser $J$ $I$ over, men $I$ begrenser ikke $J$ over,
  altså $I$ og $J$ samfaller ikke over.
\end{bemerk}

Denne relasjonen induserer en ekvivalensrelasjon på
strekkoder.

\begin{definisjon}\label{def:EkRelBarc}
  For en strekkode $\Bcal$ lar vi $(I,n)\sim_a (J,k)$ hvis
  $I\sim_a J$.
\end{definisjon}

For enhver ekvivalensklasse $e\in\Ical/\sim_a$ lar vi
$\Bcal^e$ være den korresponderende ekvivalensklassen
i $\Bcal/\sim_a$, altså $(I,n)\in\Bcal^e$ hvis $I\in e$.
Hvis det ikke eksisterer intervall i $\Bcal$ som også er
i $e$ lar vi $\Bcal^e=\emptyset$.

I \citep[seksjon 3.2]{Bauer2020} setter forfatter en total orden
på $\Bcal^e$ ved å la $(I,n) < (I,n')$ hvis $I$ strengt tatt
inneholder $J$ eller $I=J$ og $n<n'$. Dette gir det induserte
koblingsteoremet for monomorfier og epimorfier.

\begin{proposisjon}[Det induserte koblingsteoremet for monomorfier]
Hvis $f:M\to N$ er en monomorfi av varighetsmoduler, har vi at
\begin{enumerate}
  \item for enhver $e\in\Ical$,
    \[|\Bcal(M)^e|\leq|\Bcal(N)^e|.\]
    Altså, har vi en monomorfi $\chi(f):\Bcal(M)\monicto\Bcal(N)$,
    som sender det $i$-te elementet av $\Bcal(M)^e$ til det $i$-te
    elementet i $\Bcal(N)^e$.
  \item $\chi(f)$ er en monomorfi i \Barc.
\end{enumerate}
\end{proposisjon}
%Vet ikke om bevis er nødvendig?

På en lik måte for vi det induserte koblingsteoremet for
epimorfier av varighetsmoduler. Vi lar $\sim_b$ være
ekvivalensrelasjonen på $\Ical$ slik at $I\sim_b J$
samsvarer under, altså $I$ begrenser $J$ under og $J$
begrenser $I$ under. 


"
For enhver ekvivalensklasse
$e\in\Ical/\sim_b$ lar vi $\Bcal^e$ være elementene
$(I,n)\in\Bcal$ slik at $I\in e$. Vi lar $(I,n)<(J,n')$
hvis $I$ er inneholdt i $J$ eller $I=J$ og $n'<n$
"

\begin{definisjon}\label{def:zetafunktor}
  Det eksisterer en funktor $\zeta:\Barc\to\vect^\Rbld$
  definert som følger:

  Gitt en strekkode $\Ccal$ og $s\in\Rbb$ er $\zeta(\Ccal)_s$
  vektorrommet $\span_K\bset{I\in\Ccal}{s\in I}$.
  Overgangsavbildingene
  \[\p_{\zeta(\Ccal)}(s,t):\zeta(\Ccal)_s\to\zeta(\Ccal)_t\]
  for $s\leq t$, er den lineære avbildingen gitt ved
  \[I \mapsto 
  \begin{cases}
    I,\quad \text{hvis $t\in I$}\\
    0,\quad \text{ellers}
  \end{cases}
  \]
  for et basiselement $I\in\zeta(\Ccal)_s$.

  For en overlappkobling $\sigma:\Ccal\to\Dcal$ lar vi
  \[\zeta(\sigma)_s(I) =
  \begin{cases}
    J,\quad \text{Hvis $(I,J)\in\sigma$ og $s\in I\cap
    J$}\\
    0,\quad\text{ellers}
  \end{cases}.\]
\end{definisjon}
Denne funktoren gir en konvers versjon av det induserte
koblingsteoremet.

\begin{proposisjon}{break}\label{prop:KonvIKT}
  \begin{enumerate}
    \item $\zeta(\Bcal(M))\cong M$ for enhver p.e.d.
      varighetsmodul.
    \item Hvis $f:\Ccal\to\Dcal$ er en morfi i $\Barc$ med
      $\delta$-triviell (ko)kjerne, da har $\zeta(f)$
      $\delta$-triviell (ko)kjerne.
  \end{enumerate}
\end{proposisjon}
\bevis{
  \begin{enumerate}
    \item
      Vi har
      \[M_s\cong\bigoplus_{I\in\Bcal(M)}K^I_s\]
      Hvis $s\notin I$ har vi at $K^I_s = 0$, dermed har vi
      \[\bigoplus_{I\in\Bcal(M),\\ s\in I}K^I_s.\]
      Dermed er $\dim M_s$ er lik antall intervaller
      $I\in\Bcal(M)$ slik at $s\in I$. Siden
      \[\zeta(\Bcal(M))_s
      = \span_K\bset{I\in\Bcal(M)}{s\in I}\]
      er dimensjonen av $\zeta(\Bcal(M))$ også antall
      intervaller $I\in\Bcal(M)$ slik at $s\in I$. Dermed er
      $\dim\zeta(\Bcal(M))_s = \dim M_s$, ved
      \bref{teorem}{trm:DimIso} har vi at
      $\zeta(\Bcal(M))_s\cong M_s$ dermed har vi
      $\zeta(\Bcal(M))\cong M$.
    \item
      La $f:\Ccal\to\Dcal$ være en morfi i $\Barc$ med
      $\delta$-triviell kjerne. 
  \end{enumerate}

}
