\section{Det induserte koblingsteoremet}
For å relatere flaskehalsavstanden av strekkoder og
inflettingsavstanden av varighetsmoduler må vi ha en måte
å relatere en $\delta$-inflettingsmorfi av
varighetsmoduler med en $\delta$-kobling av deres
strekkoder. Det induserte koblingsteoremet lar oss
relatere varighetsmodulmorfier til en korresponderende
overlappkobling. \citep[Teorem 3.1]{Bauer2020} gir
følgende resultat

\begin{teorem}\label{trm:IKT}
  La $f: M\to N$ være en varighetsmodulmorfi på p.e.d.
  varighetsmoduler.
  \begin{enumerate}
    \item Den induserte parvis koblingen
      $\chi(f):\Bcal(M)\to\Bcal(N)$ er en overlappkobling.
    \item Hvis $\ker f$ er $\delta$-triviell, da har vi
      \begin{enumerate}
        \item for ethvert par $(I,J)\in\chi(f)$, begrenser $J$
          $I(\delta)$ over og
        \item ethvert intervall i $\Bcal(M)$ som ikke er
          koblet av $\chi(f)$ er inneholdt i et halvåpent
          intervall av lengde $\delta$.
      \end{enumerate}

    \item Hvis $\coker f$ er $\delta$-triviell, da har vi
      \begin{enumerate}
        \item for ethvert par $(I,J)\in\chi(f)$, begrenser
          $I(\delta)$ $J$ under og
        \item for ethvert intervall $\Bcal(N)$ som ikke er
          koblet av $\chi(f)$ er inneholdt i et halvåpent
          intervall av lengde $\delta$.
      \end{enumerate}


  \end{enumerate}
\end{teorem}

For å bevise dette teoremet må vi først definere hva
$\chi(f)$ er for en varighetsmodul $f: M\to N$.

\subsection{Koblinger indusert av mono- og epimorfier av
varighetsmoduler}
I \citep[seksjon 3.2]{Bauer2020} starter forfatter med
å definere $\chi(f)$ for monomorfier og epimorfier av
varighetsmoduler. Dette gjøres ved å bruke indeksene til
intervallene i strekkodene. Forfatter definerer en
ekvivalensrelasjon $\sim_a$ på mengden av intervaller
$\Ical$ ved å la $I\sim_a J$ hvis $I$ og $J$ samfaller
over, altså $I$ begrenser $J$ over og $J$ begrenser $I$
over.

\begin{bemerk}
  La $I = \langle a,b)$ og $J = \langle a,b]$, da
  begrenser $J$ $I$ over, men $I$ begrenser ikke $J$ over,
  altså $I$ og $J$ samfaller ikke over.
\end{bemerk}

Denne relasjonen induserer en ekvivalensrelasjon på
strekkoder.

\begin{definisjon}\label{def:EkRelBarc}
  For en strekkode $\Bcal$ lar vi $(I,n)\sim_a (J,k)$ hvis
  $I\sim_a J$.
\end{definisjon}

For enhver ekvivalensklasse $e\in\Ical/\sim_a$ lar vi
$\Bcal^e$ være den korresponderende ekvivalensklassen
i $\Bcal/\sim_a$, altså $(I,n)\in\Bcal^e$ hvis $I\in e$.
Hvis det ikke eksisterer intervall i $\Bcal$ som også er
i $e$ lar vi $\Bcal^e=\emptyset$.
