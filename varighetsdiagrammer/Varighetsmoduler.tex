\section{Varighetsmoduler}\label{sek:VarMod}
% Et sentralt tema for å kunne forstå stabilitet og topologisk dataanalyse er ideen om varighetsmoduler. I dette kapitellet går vi gjennom en litt abstrakt introduksjon og så ser vi på hvorfor de er viktige innenfor topologisk dataanalyse. Definisjonen på en varighetsmodul er kort og enkel.


Stabilitetsteoremet benytter to strukturer kalt
varighetsmoduler og barkoder og mål på hvor like to
varighetsmoduler og to barkoder er kalt
inflettingsavstanden mellom varighetsmoduler og
flaskehalsavstanden mellom barkoder.

Denne seksjonen er basert på \citep[seksjon 2.1]{Bauer2015}
\begin{definisjon}\label{Def:PersMod}
    En varighetsmodul $M$ er en funktor $M:\Rbld\to\vect_K$.
\end{definisjon}

Vi skriver $M_t$ for vektorromet $M(t)$. Siden en varighetsmodul $M$ er en funktor fra pomengden $\Rbld$ til $\Vect_K$ så har vi for hver $s\leq t$ en lineær avbilding $\p_M(s,t): M_s\to M_t$ som vi kaller overgangsavbildinger.

Gitt to varighetsmoduler $M$ og $N$ kan vi definere en morfi $f: M\to N$ som en samling av lineære avbildinger
$\bset{f_s: M_s\to N_s}{s\in\Rbb}$ slik at diagrammet
\[
\begin{tikzcd}
	{M_s} && {N_s} \\
	\\
	{M_t} && {N_t}
	\arrow["{f(s)}", from=1-1, to=1-3]
	\arrow["{\p_M(s,t)}", from=1-1, to=3-1]
	\arrow["{\p_N(s,t)}", from=1-3, to=3-3]
	\arrow["{f(t)}", from=3-1, to=3-3]
\end{tikzcd}
\]
kommuterer. Vi kan komponere morfiene på den følgende måten; gitt morfier $f:M\to N$ og $g:M\to P$ er $g\circ f$ definert som samlingen $\{g_s\circ f_s: M_s\to P_s\;|\;s\in\Rbb\}$.

Siden vi har objekter, varighetsmoduler og vi har morfier mellom dem kan vi definere kategorien av varighetsmoduler

\begin{definisjon}\label{Def:KatPMod}
    Kategorien $\Vect_K^\Rbld$ er kategorien av varighetsmodulene med varighetsmodul-morfier mellom dem.
\end{definisjon}

Som notasjonen foreslår er kategorien av varighetsmoduler en funktorkategori.
Et enkelt, men viktig eksempel på en varighetsmodul er intervallmodulen

\begin{eksempel}\label{eks:IntMod}
For et intervall $I\subset\Rbb$ eksisterer det en varighetsmodul $K^I$ definert på følgende måte i \cite{Bauer2020}
\[
K^I_r =
\begin{cases}
K, \quad\text{hvis } r\in I\\
0, \quad\text{ellers.}
\end{cases}\quad 
\p_{K^I}(r,s) = 
\begin{cases}
\id_K,\quad r,s\in I\\
0,\quad \text{ellers}
\end{cases}
\]
\end{eksempel}

\begin{definisjon}\label{prop:VMDS}
  For varighetsmoduler $M$ og $N$ er den direkte summen
  deres definert punktvis
\end{definisjon}

\begin{eksempel}\label{eks:varighetsmodul}
  For intervaller $(6,8]$, $[7,9]$, $[1,2)$, $(4,10)$ og
  $(1,2]$ har vi varighetsmodulen 
  \[M = K^{(6,8]}\oplus
  K^{[7,9]}\oplus K^{[1,2)}\oplus K^{(4,10)}\oplus
  K^{(1,2]}.\]
  I $t = 1,5$ har vi $M_t = K\oplus K$ siden $t\in[1,2)$
  og $t\in (1,2]$. I $t=5$ er $M_t = K$ siden $t\in
  (4,10)$ og ingen andre intervaller. I $t=7,5$ er $M_t
  = K\oplus K\oplus K$ siden $t\in (4,10)$, $t\in [7,9]$
  og $t\in(6,8]$.
\end{eksempel}

I denne kategorien kan en varighetsmodul inneholde
uendeligdimensjonale vektorrom. Fra \citep[teorem
2.8]{Chazal2016} kan disse ikke bli
intervalldekomponert. Dette er et problem når vi skal
studere barkodenediagrammene til varighetsmoduler.
I denne oppgaven begrenser vi oss bare til
varighetsmoduler hvor alle vektorrommene er
endeligdimensjonale. Varighetsmoduler der vektorromene er alle
endeligdimensjonale er kalt punktvis endeligdimensjonale
varighetsmoduler eller p.e.d. varighetsmoduler tatt fra
\cite{Bauer2015}. Kategorien av endeligdimensjonale
vektorrom over en kropp $K$ skriver vi som $\vect_K$,
i motsetning til kategorien av alle vektorrom skrevet
$\Vect_K$ og kategorien av p.e.d. varighetsmoduler er da skrevet
$\vect_K^\Rbld$.

\subsection{Inflettingsavstand}
Stabilitet av varighetsmoduler innebærer relasjonen mellom
to typer avstander, flaskehalsavstanden mellom barkoder
og inflettingsavstanden mellom varighetsmoduler. Her
definerer vi inflettingsavstanden mellom to
varighetsmoduler.

For å definere distansen må vi gjennom noen få steg.
I \citep[seksjon 3.1]{Bauer2015} gir forfatter følgende
definisjoner for forskyvingsfunktorene, inflettinger,
inflettingsavstanden og en utvidet pseudometrikk.
\begin{definisjon}\label{Def:DShift}
	En $\delta$-forskyvning av en varighetsmodul er en funktor
	\[(\cdot)(\delta): \Pers\to \Pers\]
	Som tar en varighetsmodul $M$ til $M(\delta)$ hvor
  $M(\delta)_t = M_{t+\delta}$ og tar
  varighetsmodulmorfier $f:M\to N$ til
  $f(\delta):M(\delta)\to N(\delta)$ gitt ved
  $f(\delta)_t=f_{t+\delta}$.
\end{definisjon}

\begin{lemma}\label{lem:DSShiftSplit}
  For varighetsmoduler $M$ og $N$ og for $\delta\geq0$ er
  \[(M\oplus N)(\delta) = M(\delta)\oplus N(\delta)\]
\end{lemma}
\bevis{
  La $s\in\Rbb$ da er $(M\oplus N)(\delta)_s = (M\oplus
  N)_{s+\delta}$ fra \bref{definisjon}{def:DS} er
  $(M\oplus N)_{s+\delta} = M_{s+\delta}\oplus
  N_{s+\delta}$ som er lik $M(\delta)_s\oplus
  N(\delta)_s$. Siden varighetsmoduler er isomorfe hvis de
  er punktvis isomorfe har vi
  \[(M\oplus N)(\delta) = M(\delta)\oplus N(\delta)\]
}

\begin{proposisjon}\label{kor:Shiftpluss}
For $\delta,\delta'\in\Rbb$ er
  $((\cdot)(\delta))(\delta')=(\cdot)(\delta+\delta')$
\end{proposisjon}
\bevis{
For en varighetsmodul $M$ er
  $(M(\delta))(\delta')_t=M(\delta)_{t+\delta'}=M_{t+\delta+\delta'}=M(\delta+\delta')_t$
  for enhver $t\in\Rbb$. For samme grunn er
  $(f(\delta))(\delta')=f(\delta+\delta'
)$ for enhver varighetsmodulmorfi.
}

Denne funktoren gir oss konseptet av en $\delta$-infletting.

\begin{definisjon}\label{Def:infletting}
	La $M$ og $N$ være varighetsmoduler. Vi sier at $M$ og
  $N$ er $\delta$-inflettet hvis det eksisterer
  varighetsmodulmorfier $f:M\to N(\delta)$ og $g:N\to
  M(\delta)$ slik at
	\[g(\delta)\circ f = \p_M(t,t+2\delta),\quad
  f(\delta)\circ g = \p_N(t,t+2\delta).\]
\end{definisjon}
Vi skriver $\p^\eps_M(t) = \p_M(t,t+\eps)$. Bemerk at
$\p^0_M=\id_M$ fordi $\p^0_M(t)
= \p_M(t,t+0)=\p_M(t,t)=\id_M$. Vi kaller
varighetsmodulmorfiene $f$ og $g$ for
$\delta$-inflettingsmorfiene.

\begin{definisjon}\label{def:UtPsMet}
   En utvidet pseudometrikk på en klasse $X$ er en
   funksjon $d: X\times X\to [0,\infty]$ med følgende
   egenskaper:
   \begin{itemize}
     \item $d(x,x) = 0$ for enhver $x\in X$,
     \item $d(x,y) = d(y,x)$ for enhver $x,y\in X$
     \item $d(x,z) \leq d(x,y) + d(y,x)$ for enhver
       $x,y,z\in X$ slik at $d(x,y),d(y,z)\leq\infty$.
   \end{itemize}
\end{definisjon}

Forskjellen på en utvidet pseudometrikk og en metrikk er
at det kan være to elementer $x,y\in X$ slik at $x\neq y$,
men $d(x,y)=0$ i motsetning til metrikker der $d(x,y)=0$
hvis og bare hvis $x=y$.

\begin{definisjon}\label{Def:InfAvs}
	For $M$ og $N$ varighetsmoduler definerer vi inflettingsavstanden $d_I$ ved
	\[d_I(M,N) = \inf\{\delta\in \ropen{0}{\infty}\mid \text{$M$ og $N$ er $\delta$-inflettet}\}\]
\end{definisjon}
Denne avstanden gir et tall på hvor "isomorfe" to varighetsmoduler er.

Vi går gjennom et lemma gitt i \cite{Bauer2015} som en bemerkning (\cite[Remark 3.1]{Bauer2015}).

\begin{lemma}\label{lem:Inf_Sum}
La $L$, $M$ og $N$ være varighetsmoduler slik at $L$ og
  $M$ er $\delta$-inflettet og $M$ og $N$ er
  $\delta'$-inflettet. Da er $L$ og $N$
  $(\delta+\delta')$-inflettet.
\end{lemma}
\bevis{
La $f: L\to M(\delta)$, $g: M\to L(\delta)$ være $\delta$
inflettingsmorfier for $L$ og $M$, la $h: M\to N(\delta')$
og $i: N\to M(\delta')$ være inflettingsmorfier for $M$ og
$N$. Vi viser at $F = h(\delta)\circ f:L\to
N(\delta+\delta')$ og $G=g(\delta')\circ i: N\to
L(\delta+\delta')$ er
$(\delta+\delta')$-inflettingsmorfier for $L$ og $N$. Vi
har komposisjonen
\[G(\delta+\delta')\circ F = (g(\delta')\circ
i)(\delta+\delta')\circ h(\delta)\circ f.\]
Ved funktorialitet av $(\cdot)(\delta)$ får vi
\[(g(\delta')\circ i)(\delta+\delta')\circ h(\delta)\circ
f = g(\delta')(\delta+\delta')\circ i(\delta+\delta')\circ
h(\delta)\circ f.\]
Videre ved \bref{korollar}{kor:Shiftpluss} og
funktorialitet blir komposisjonen
\[g(\delta')(\delta+\delta')\circ i(\delta+\delta')\circ
h(\delta)\circ f= g(\delta+2\delta')\circ (i(\delta')\circ
h)(\delta)\circ f.\]
Nå har vi komposisjonen $i(\delta')\circ h$ som vi vet er
$\p^{2\delta'}_M$ da har vi
\[g(\delta+2\delta')\circ (i(\delta')\circ h)(\delta)\circ
f=g(\delta+2\delta')\circ\p_M^{2\delta'}\circ f.\]
Fra definisjonen av en varighetsmodulmorfi har vi at
$\p_M^{2\delta'}\circ f=f(2\delta')\circ\p_L^{2\delta'}$,
dermed får vi
\[g(\delta+2\delta')\circ\p_M^{2\delta'}\circ
f=g(\delta+2\delta')\circ f(2\delta')\circ
\p_M^{2\delta'}\]
Tilslutt får vi da
\[g(\delta+2\delta')\circ f(2\delta')\circ
\p_L^{2\delta'}=(g(\delta)\circ
f)(2\delta')\circ\p_L^{2\delta'}
= \p_L^{2\delta}(2\delta')\circ \p_L^{2\delta'}\]
\[=\p_L^{2\delta+2\delta'}\circ
\p_L^{2\delta'}=\p_L^{2\delta'+2\delta}.\]
På samme måte har vi at $F(\delta+\delta')\circ
G=\p_N^{2\delta+2\delta'}$.
}

\begin{teorem}\label{trm:Utvidet-Metrikk}
  Inflettingsavstanden er en utvidet metrikk
\end{teorem}
\bevis{
    Vi starter med første punktet
    i \bref{definisjon}{def:UtPsMet}. La $M$ være en
    varighetsmodul. Vi har en $0$-infletting $f,g:M\to
    M(0)$ gitt ved $f_t,g_t = \p(t,t+0)=\id_{M_t}$. Siden
    vi ikke kan finne en $\delta$ mindre er $d_I(M,M)=0$.
    
    Varighetsmoduler $M$ og $N$ er $\delta$-inflettet hvis
    og bare hvis $N$ og $M$ er $\delta$-inflettet. Dette
    betyr at $d_I(M,N)=d_I(N,M)$.

    La $L$, $M$ og $N$ være varighetsmoduler slik at $L$
    og $M$ er $\delta$-inflettet og $M$ og $N$ er
    $\delta'$-inflettet. Da er $L$ og $N$
    $(\delta+\delta')$-inflettet. Ved definisjonen av
    $d_I$ har vi $d_I(L,M)\leq\delta$ og
    $d_I(M,N)\leq\delta'$, dermed får vi $d_I(L,N)\leq
    d_I(L,M)+d_I(M,N)\leq \delta+\delta'$.
}

\begin{lemma}\label{lem:d_I-Iso}
  Hvis $M$ og $M'$ er isomorfe er $d_I(M,M')=0$
\end{lemma}
\bevis{
  Siden $M\cong M'$ har vi isomorfier $f: M\to M'$ og $g:
  M'\to M$. Dette gir
  \[g\circ f = g(0)\circ f = \id_M\quad\text{og}\quad
  f\circ g = f(0)\circ g = \id_{M'}.\]
  Siden $\p_M^0 = \id_M$ og $\p_{M'}^0=\id_{M'}$ er $f$ og
  $g$ $0$-inflettingsmorfier. Dermed er $d_I(M,M')=0$.
}

\begin{proposisjon}\label{prop:InfAvs_Metrikk}
  Hvis $M\cong M'$ er isomorfe varighetsmoduler så er
  $d_I(M,N) = d_I(M',N)$ for enhver varighetsmodul.
\end{proposisjon}
\bevis{
  Siden $d_I$ er en utvidet metrikk betyr det at den
  oppfyller den trekantulikheten $d_I(M,N)\leq
  d_I(M,M')+d_I(M',N)$. Ved
  \bref{lemma}{lem:d_I-Iso} er $d_I(M,M')=0$ altså
  er $d_I(M,N)\leq d_I(M',N)$. Ved samme argument er
  $d_I(M',N)\leq d_I(M,N)$. Dermed er
  $d_I(M,N)=d_I(M',N)$.
}

Resultatet fra \bref{proposisjon}{prop:InfAvs_Metrikk}
betyr at inflettingsavstanden er en metrikk på
isomorfiklassene av varighetsmodulene som forfatter
i \cite{Bauer2015} nevner.

